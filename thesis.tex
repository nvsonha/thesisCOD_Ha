% Document class / Dokumentenklasse:
\documentclass{CODthesis}
% Parameters of the CODthesis class:
% DIV12       = document layout: DIV factor = 12 (larger number creates larger pages)
% BCOR12mm    = binding correction: 12mm
% headsepline = separate page head by a line
% twoside     = twosided document
% 11pt        = font size 11 point
% openright   = start new chapters only on right pages (odd numbered pages)
% more information: http://tug.ctan.org/tex-archive/macros/latex/contrib/koma-script/scrguien.pdf

% Packages used in LNTthesis class:
% package[english]{babel}   % english language / Englische Sprache
% package{LNTthesis}        % LNT specific definitions / LNT spezifische Definitionen
% package{graphicx}         % for using eps images / Einbinden von EPS Grafiken
% package{verbatim}         % for quickly commenting out large parts of your text / Um viel Text schnell auskommentieren zu koennen
% package{amssymb}          % additional math symbols / Zusaetzliche mathematische Symbole
% package{amsmath}          % additional math commands / Zusaetzliche mathematische Befehle
% package{amsxtra}          % even more math symbols / Noch mehr mathematische Symbole
% package{amsthm}           % theorem environment etc / Theorem Umgebung usw
% more information on amsmath: http://www.ctan.org/get/macros/latex/required/amslatex/math/amsldoc.pdf

% package{psfrag}           % psfrag: http://www.ctan.org/get/macros/latex/contrib/psfrag/pfgguide.pdf
% package{subfigure}        % enable subfigures / Ermoeglicht Subfigures (mehrere Figures neben/untereinander)

% package{tabularx}         % Advanced tabular environments with tabularx

% !! PLEASE READ THE LATEX HELP IF YOU HAVE ANY QUESTIONS !!
% http://tobi.oetiker.ch/lshort/lshort.pdf


% Macros:
    \newcommand{\eq}[1]{Equation~(\ref{#1})}        % \eg{eq:golomb}  --> Equation (2.15)
    \newcommand{\eref}[1]{(\ref{#1})}               % \eg{eq:golomb}  --> (2.15)
    \newcommand{\fig}[1]{Figure~\ref{#1}}           % \fig{fig:golomb}--> Figure 2.15
    \newcommand{\tab}[1]{Table~\ref{#1}}            % \tab{tab:lala}  --> Table 2.15
    \newtheorem{prop}{Proposition}

% Abbreviations
    \newcommand{\equivalent}{\triangleq}
    \newcommand{\given}{\:\!\vert\:\!}

\usepackage[T1]{fontenc}
\usepackage[latin9]{inputenc}
\usepackage{array}
\usepackage{varioref}
\usepackage{float}
\usepackage{rotfloat}
\usepackage{calc}
\usepackage{amsmath}
\usepackage{amsthm}
\usepackage{amssymb}
\usepackage{stackrel}
\usepackage{wasysym}
\PassOptionsToPackage{normalem}{ulem}
\usepackage{ulem}
\usepackage{textcomp}
\usepackage{lmodern}

\makeatletter

%%%%%%%%%%%%%%%%%%%%%%%%%%%%%% LyX specific LaTeX commands.
\pdfpageheight\paperheight
\pdfpagewidth\paperwidth

%% Because html converters don't know tabularnewline
\providecommand{\tabularnewline}{\\}
\floatstyle{ruled}
\newfloat{algorithm}{tbp}{loa}
\providecommand{\algorithmname}{Algorithm}
\floatname{algorithm}{\protect\algorithmname}

%%%%%%%%%%%%%%%%%%%%%%%%%%%%%% Textclass specific LaTeX commands.
\theoremstyle{definition}
    \ifx\thechapter\undefined
      \newtheorem{defn}{\protect\definitionname}
    \else
      \newtheorem{defn}{\protect\definitionname}[chapter]
    \fi
\theoremstyle{plain}
    \ifx\thechapter\undefined
	    \newtheorem{thm}{\protect\theoremname}
	  \else
      \newtheorem{thm}{\protect\theoremname}[chapter]
    \fi
\theoremstyle{remark}
    \ifx\thechapter\undefined
      \newtheorem{rem}{\protect\remarkname}
    \else
      \newtheorem{rem}{\protect\remarkname}[chapter]
    \fi
\theoremstyle{plain}
    \ifx\thechapter\undefined
  \newtheorem{cor}{\protect\corollaryname}
\else
      \newtheorem{cor}{\protect\corollaryname}[chapter]
    \fi
\theoremstyle{definition}
    \ifx\thechapter\undefined
      \newtheorem{example}{\protect\examplename}
    \else
      \newtheorem{example}{\protect\examplename}[chapter]
    \fi
\theoremstyle{plain}
    \ifx\thechapter\undefined
      \newtheorem{lem}{\protect\lemmaname}
    \else
      \newtheorem{lem}{\protect\lemmaname}[chapter]
    \fi
\theoremstyle{plain}
    \ifx\thechapter\undefined
      \newtheorem{conjecture}{\protect\conjecturename}
    \else
      \newtheorem{conjecture}{\protect\conjecturename}[chapter]
    \fi

%%%%%%%%%%%%%%%%%%%%%%%%%%%%%% User specified LaTeX commands.
\usepackage{placeins}
\usepackage{graphicx}
\usepackage{algorithmic}
\usepackage{listings}

\makeatother

\usepackage{babel}
\providecommand{\conjecturename}{Conjecture}
\providecommand{\corollaryname}{Corollary}
\providecommand{\definitionname}{Definition}
\providecommand{\examplename}{Example}
\providecommand{\lemmaname}{Lemma}
\providecommand{\remarkname}{Remark}
\providecommand{\theoremname}{Theorem}


% Document:
\begin{document}

% ###################################
% Title page / Titelseite
% ###################################
\CODtitle{Master's Thesis}          % Thesis type / Art der Arbeit (Master's Thesis, Diplomarbeit)
    {Vector Network Coding Gap Sizes for the Generalized Combination Network}                  % Thesis title / Titel der Arbeit
    {Ha Nguyen}                  % Your name / Name des Diplomanden
    {Sven Puchinger}            % Advisor / Betreuer
    {M\"unchen, 09 2019}          % Munich, Date / Muenchen, Datum
    {Ha Nguyen\\                 % Your Address / Anschrift des Diplomanden
    Sintperstr. 50\\
    81539 M\"unchen\\
    ha.nguyen@tum.de}

\CODrecht{Ha Nguyen}             % Your name / Name des Studenten
    {M\"unchen, 05.09.2019}         % Munich, Date / Muenchen, Datum

\cleardoubleemptypage   % start new double page / neue Doppelseite

% ########################################
% Table of Contents / Inhaltsverzeichnis
% ########################################

% roman page numbering, starting with page number 1 / roemische Seitennummerierung beginnend mit Seite 1
    \setcounter{page}{1}
    \pagenumbering{roman}

        \tableofcontents    % Table of contents / Inhaltsverzeichnis
        \listoffigures      % List of figures / Abbildungsverzeichnis
        \listoftables       % List of tables / Tabellenverzeichnis
        \cleardoubleemptypage   % start new double page / neue Doppelseite

% ########################################
% Chapters / Kapitel:
% ########################################

% arabic page numbering, starting with page number 1 / arabische Seitennummerierung beginnend mit Seite 1
    \setcounter{page}{1}
    \pagenumbering{arabic}

	%%%%%%%%%%%%%%%%%%%%%%%%%%%%%%%%%%%%%%%%%%%%%%%
\chapter{Introduction} \label{chap:introduction}
%%%%%%%%%%%%%%%%%%%%%%%%%%%%%%%%%%%%%%%%%%%%%%%

\textit{Network coding} was first introduced in Ahlswede et al.'s
seminal paper \cite{Ahlswede:2000}. Network coding gives a huge advantage
in increasing throughput of information transmission in comparison
with simple routing methods for a communication network. This throughput
gain is achieved in network coding, since nodes in a network are allowed
to forward a \textit{function} of their received packets, while in
routing such packets can only be forwarded to another node. K\"otter
and M\'edard provided an algebraic formulation for a \textit{linear
network coding} problem and its scalar solvability. If functions on
all nodes are linear, then we obtain a linear network coding solution,
and a \textit{solution} is an assignment of all functions such that
all destination nodes can recover all of their requested messages
transmitted from a source node. Network coding was further developed
with \textit{vector network coding} by Ebrahimi \cite{Ebrahimi:2011},
where all packets are vectors of length $t$. In \cite{Wachter-Zeh:2018},
Etzion and Wachter-Zeh proved that vector network coding based on
subspace codes outperforms linear network coding. It gives the motivation
for our study in this thesis on vector network coding, especially
the study of \textit{gap} measuring the difference in \textit{alphabet
sizes} between solutions of scalar and vector network coding. The
alphabet size is an important parameter determining the amount of
computation performed at each network node \cite{Wachter-Zeh:2018}.
The problem of finding the minimum required alphabet size of a (linear
or nonlinear) scalar network code for a certain multicast network
is NP-complete \cite{Langberg:2009,Lehman:2004,Gone:2018}. 

\textit{Generalized combination networks} (GCN) were first introduced
in \cite{Wachter-Zeh:2018}, where they have shown many instances
of the network with vector solutions outperforming scalar solutions.
However, no general vector solution was found a generalized combination
network with 3 source messages. It motivates us to study it in this
study, together with the gap for such a solution. We further develope
the study on gap for 2 families of GCN by both combinatorial approaches
and computational approaches. 

\textbf{Outline}

In \textbf{Chapter 2}, we first recall coding-theory-specific notions
and give an introduction to the known codes that we consider in this
thesis. Then, we give the definition of \textit{maximum rank distance}
(MRD) code and its properties. This code was mainly used to study
vector solutions for multiple families of the \textit{generalized
combination network} (GCN) in \cite{Wachter-Zeh:2018}. Finally, we
give the definition of Grassmannian code, Covering Grassmannian code,
Multiple Grassmannian code and the notion of the maximum size of a
Multiple Grammanian code. Because Grassmannian codes contain subspaces
of the same dimension over a finite field $\ensuremath{\mathbb{F}}_{q}$,
they have been recently applied in the study of network coding problems,
such as \cite{Etzion:2016,Etzion:2018,Wachter-Zeh:2018,Zhang:2019}.

The remaining chapters contain our study of new gap sizes for GCN.
We divided them into three main parts: Chapter 3 and 4 explain how
we transfer vector network coding problems for GCN to problems of
finding matrices or Grassmanian codes, Chapter 5 contains new gap
sizes for 2 families of GCN with combinatorial proofs based on Lov\'asz
Local Lemma (LLL), and Chapter 6 contains new computational results
of vector solutions outperforming scalar solutions for the $\left(\epsilon=1,\ell=1\right)-\mathcal{N}_{h=3,r,s=4}$
network. The details of each chapter are mentioned below.

In \textbf{Chapter 3} and \textbf{Chapter 4}, we represent network
as a matrix channel and introduce an advantage of vector solutions
in alphabet sizes in comparison with scalar solutions for network
coding problems. We firstly recall the motivation of network coding,
and secondly we introduce how we approach such problems by fomulating
the relationship between source's messages and receiver's packets
by linear equation systems. Thirdly, we explain why we choose GCN
for our study, and we recall separately known theorems on an existence
of a scalar or a vector solution for GCN. Finally, we formulate the
\textit{gap} to measure the difference in alphabet sizes between a
vector solution and a corresponding optimal scalar solution. We list
known gaps for some instances of GCN in previous studies together
with our new found gaps in this study.

In \textbf{Chapter 5}, we present new gaps found by an approach based
on LLL. We start this chapter with a simple network $\left(\epsilon=1,\ell=1\right)-\mathcal{N}_{h=3,r,s=4}$.
No general vector solution outperforming scalar network coding was
found in previous studies. The gap of the $\left(\epsilon=1,\ell=1\right)-\mathcal{N}_{h=3,r,s=4}$
network for such purpose is found with combinatorial proofs in this
study. During the proof of the gap, we also prove there always exists
vector solutions for the network, if and only if the number $r$ of
intermediate nodes is less than or equal to a certain number. After
achieving the gap for the $\left(\epsilon=1,\ell=1\right)-\mathcal{N}_{h=3,r,s=4}$
network, we develop the proofs further for the $\left(\epsilon=1,\ell=1\right)-\mathcal{N}_{h,r,s}$
network and the $\left(\epsilon=1,\ell>1\right)-\mathcal{N}_{2\ell,r,2\ell+1}$
network. Knowing the gap motivates us to search for vector solutions
achieving such gap, which leads to computational results presented
in Chapter 6.

\textbf{Chapter 6} shows the core steps of 4 different computational
approaches to find a vector solution outperforming the optimal scalar
solution for the $\left(\epsilon=1,\ell=1\right)-\mathcal{N}_{h=3,r,s=4}$
with $t=2$ and $t=3$. We have found a vector solution of 89 nodes
which is roughly two times 42 nodes of the optimal scalar solution
for the network with $t=2$. For the network with $t=3$, a vector
solution of 166 nodes is found when the optimal scalar solution of
such network exists if and only the number of nodes is less than or
equal to 146. We then conclude the new bound on maximum size of Grassmanian
codes for such network respectively, $89\leq\mathcal{A}_{2}\left(6,4,3;2\right)\leq126$
and $166\leq\mathcal{A}_{2}\left(9,6,3;2\right)\leq537$. While writing
this thesis, the bound of the code when $t=2$ has been improved in
\cite{Etzion:2018}. The result of $t=3$ have not yet been found
in our scope of knowledge. 

\clearpage

	%%%%%%%%%%%%%%%%%%%%%%%%%%%%%%%%%%%%%%%%%%%%%%%
\chapter{Preliminaries} \label{chap:preliminaries}
%%%%%%%%%%%%%%%%%%%%%%%%%%%%%%%%%%%%%%%%%%%%%%%

\section{Notation and Basic Terminology}

\paragraph{Vectors and Matrices}

Vectors $\boldsymbol{v}$ are denoted by underlined letters. Unless
stated otherwise, vectors are indexed starting from 1, i.e. $\boldsymbol{v}=\left[v_{1},\ldots,v_{n}\right]$.
Vectors are usually considered to be row vectors. Matrices $\boldsymbol{V}$
are shown in bold and capital letters. Elements of matrices or vectors
are surrounded by square brackets, and elements of tuples are surrounded
by round brackets.

\paragraph{Vector space}

A vector space of dimension $n$ over a finite field with $q$ elements
is denoted by $\ensuremath{\mathbb{F}}_{q}^{n}$. 

\paragraph{Linear Block Code}

>\textcompwordmark >\textcompwordmark > WHY ONLY LINEAR?

\textit{Block codes} are finite length codes first studied by Golay
\cite{Golay:1949} and Hamming \cite{Hamming:1950}. A \textit{linear}
$\left[n,k\right]_{q}$ \textit{block code} $\ensuremath{\mathcal{C}}$
of length $n$, dimension $k$ and codimension (or redundancy) $r=n-k$,
is a $k$-dimensional \textit{linear subspace} of $\ensuremath{\mathbb{F}}_{q}^{n}$.
A \textit{generator matrix} $\boldsymbol{G}$ has $k$ rows, which
form a \textit{basis} of $\ensuremath{\mathcal{C}}$, i.e., $\boldsymbol{G}$'s
row space is a set of $k$ linearly independent vectors generating
$\ensuremath{\mathcal{C}}$. The codeword $\underline{c}\in\ensuremath{\mathcal{C}}$
is encoded by the \textit{information vector} (or \textit{message})
$\boldsymbol{m}$, with $\boldsymbol{c}=\boldsymbol{m}\cdot\boldsymbol{G}$.
For any $\boldsymbol{c}\in\ensuremath{\mathcal{C}}$: $\boldsymbol{c}\cdot\boldsymbol{H}^{T}=\boldsymbol{0}$,
with $\boldsymbol{H}$ is a \textit{parity-check matrix} of $\ensuremath{\mathcal{C}}$,
whose row space generates the $\left[n,n-k\right]_{q}$ dual code
$\ensuremath{\mathcal{C}}^{\perp}$. $\left[n,k,d\right]_{q}$ is
equivalent to $\left[n,k\right]_{q}$ with minimum distance $d$ fulfills
$d=\underset{\boldsymbol{c}\in\ensuremath{\mathcal{C}\setminus\left\{ \boldsymbol{0}\right\} }}{min}d\left(\boldsymbol{c},\boldsymbol{0}\right)$
with respect to a metric $d\left(\cdot,\cdot\right):\ensuremath{\mathbb{F}}^{n}\times\ensuremath{\mathbb{F}}^{n}\rightarrow\mathbb{R}_{\geq0}$
on $\ensuremath{\mathbb{F}}^{n}$ \footnote{i.e., $d\left(\boldsymbol{x},\boldsymbol{y}\right)\geq0$,$d\left(\boldsymbol{x},\boldsymbol{y}\right)=0$
iff $\boldsymbol{x}=\boldsymbol{y}$, $d\left(\boldsymbol{x},\boldsymbol{y}\right)=d\left(\boldsymbol{y},\boldsymbol{x}\right)$
and $d\left(\boldsymbol{x},\boldsymbol{z}\right)\leq d\left(\boldsymbol{x},\boldsymbol{y}\right)+d\left(\boldsymbol{y},\boldsymbol{z}\right)$
for all $\boldsymbol{x},\boldsymbol{y},\boldsymbol{z}\in\ensuremath{\mathbb{F}}^{n}$.}.

\paragraph{Gaussian coefficient}

Gaussian coefficient (also known as $q$-binomial) counts the number
of subspaces of dimension $k$ in a vector space $\ensuremath{\mathbb{F}}_{q}^{n}$,

\[
\left[\begin{array}{c}
n\\
k
\end{array}\right]_{q}=\stackrel[i=0]{k-1}{\prod}\frac{q^{n}-q^{i}}{q^{k}-q^{i}}
\]


\paragraph{Multigraph}

A graph is permitted to have multiple edges. Edges that are incident
to same vertices can be in parallel. 

\paragraph{Directed Acyclic Graph}

A finite directed graph with no directed cycles, i.e. it consists
of a finite number vertices and edges, with each edge directed from
a vertex to another, such that there is no loop from any vertex $v$
with a sequence of directed edges back to the vertex again $v$.

\paragraph{Multicast}

Multicast communication supports the distribution of a data packet
to a group of users \cite{Zhang:2012}. It can be one-to-many or many-to-many
distribution \cite{Harte:2008}. In this study, we consider only one-to-many
multicast network.

\paragraph{Asymptotic Behavior}

For the combinatorial results, we study the asymtotic behaviour of
some formulas depending on the alphabet size $q$ and the vector length
$t$, by using the Bachmann-Landau notation, i.e. $\mathcal{O}\left(f\left(q,t\right)\right)$
for upper, $\Theta\left(f\left(q,t\right)\right)$ for tight, and
$\Omega\left(f\left(q,t\right)\right)$ for lower bounds, where $f$
is a function of the alphabet size and the vector length.

\section{Definition}

>\textcompwordmark >\textcompwordmark > EXPLAIN WHY I NEED EACH
DEFINITION?
\begin{defn}[Grassmannian Code]
 A Grassmannian code is a set of all subspaces of dimension $k\leq n$
in $\ensuremath{\mathbb{F}}_{q}^{n}$, and is denoted by $\mathcal{G}_{q}\left(n,k\right)$.
Due to being the set of all subspaces that have the same dimension
$k$, it is also called a \textit{constant dimension code}. \cite{Zhang:2019}
\end{defn}
%
\begin{defn}[Projective Space]
 The \textit{projective space of order} $n$ is a set of all subspaces
of $\ensuremath{\mathbb{F}}_{q}^{n}$, and is denoted by $\mathcal{P}_{q}\left(n\right)$,
i.e. a union of all dimension $k=0,\ldots n$ subspaces in $\ensuremath{\mathbb{F}}_{q}^{n}$
or $\mathcal{P}_{q}\left(n\right)=\bigcup_{k=0}^{n}\mathcal{G}_{q}\left(n,k\right)$.
\cite{Wachter-Zeh:2018}
\end{defn}
%
\begin{defn}[Covering Grassmannian code]
 An $\alpha-\left(n,k,\delta\right)_{q}^{c}$ covering Grassmannian
code (code in short) $\mathcal{C}$ is a subset of $\mathcal{G}_{q}\left(n,k\right)$
such that each subset of $\alpha$ codewords of $\mathcal{C}$ span
a subspace whose dimension is at least $\delta+k$ in $\ensuremath{\mathbb{F}}_{q}^{n}$.
\cite{Zhang:2019}
\end{defn}

\paragraph{The cardinality of a Grassmannian code}

The cardinality of $\mathcal{G}_{q}\left(n,k\right)$ is the Gaussian
coefficient (also known as $q$-binomial), which counts the number
of subspaces of dimension $k$ in a vector space $\ensuremath{\mathbb{F}}_{q}^{n}$,

\[
\left|\mathcal{G}_{q}\left(n,k\right)\right|=\left[\begin{array}{c}
n\\
k
\end{array}\right]_{q}=\stackrel[i=0]{k-1}{\prod}\frac{q^{n}-q^{i}}{q^{k}-q^{i}},
\]

where $q^{\left(n-k\right)k}\leq\left[\begin{array}{c}
n\\
k
\end{array}\right]_{q}\leq4q^{\left(n-k\right)k}$.

\begin{defn}[Subspace packing \cite{Etzion:2018}]
 A subspace packing $t-\left(n,k,\lambda\right)_{q}^{m}$ is a set
$\mathcal{S}$ of $k$-subspaces or $k$-dimensional subspaces (called
\textit{blocks}), such that each $t$-subspace of $\ensuremath{\mathbb{F}}_{q}^{n}$
is contained in at most $\lambda$ codewords of $\mathcal{C}$. 
\end{defn}
%
\begin{defn}[\cite{Etzion:2018}]
$\mathcal{A}_{q}\left(n,k,t;\lambda\right)$ denotes the maximum
size of a $t-\left(n,k,\lambda\right)_{q}^{m}$ code, where there
are no repeated codewords. 
\end{defn}

\section{Theorem}

>\textcompwordmark >\textcompwordmark > EXPLAIN BEFORE WHY I NEED
EACH THEOREM?
\begin{thm}[\cite{Zhang:2019}]
 If $n,k,t,$ and $\lambda$ are positive integers such that $1\leq t<k<n$
and $1\leq\lambda\leq\left[\begin{array}{c}
n-t\\
k-t
\end{array}\right]_{q}$, then

\[
\mathcal{A}_{q}\left(n,k,t;\lambda\right)\leq\left\lfloor \lambda\frac{\left[\begin{array}{c}
n\\
t
\end{array}\right]_{q}}{\left[\begin{array}{c}
k\\
t
\end{array}\right]_{q}}\right\rfloor 
\]
\end{thm}
%
\begin{thm}[\cite{Zhang:2019}]
 If $n,k,t,$ and $\lambda$ are positive integers such that $1\leq t<k<n$
and $1\leq\lambda\leq\left[\begin{array}{c}
n-t\\
k-t
\end{array}\right]_{q}$, then

\[
\mathcal{A}_{q}\left(n,k,t;\lambda\right)\leq\left\lfloor \frac{q^{n}-1}{q^{k}-1}\mathcal{A}_{q}\left(n-1,k-1,t-1;\lambda\right)\right\rfloor 
\]
\end{thm}

\clearpage
	%%%%%%%%%%%%%%%%%%%%%%%%%%%%%%%%%%%%%%%%%%%%%%%
\chapter{Network Coding} \label{chap:network_coding}
%%%%%%%%%%%%%%%%%%%%%%%%%%%%%%%%%%%%%%%%%%%%%%%

\section{What is Network Coding? \label{sec:What-is-NC}}

We first explain the term of network, then we futher introduce Network
Coding. Our considered communication network is a directed graph\footnote{Network coding over undirected networks was introduced in \cite{Li:2004}.}
allowing multiple links from one node to another. Each \textit{link}
in a network has \textit{unit capacity}, i.e. it carries a packet
which is either a symbol from $\ensuremath{\mathbb{F}}_{q_{\mathrm{s}}}$,
or a vector of length $t$ over $\ensuremath{\mathbb{F}}_{q}$. Note
that the assumption of unit capacity does not restrict our considered
networks in Section \ref{sec:Description_GCN}, since links of larger
capacity can be represented by multiple parallel links $\ell$ of
a data unit. In Figure \ref{fig:incoming_links}(a), a node of a network
is represented with its \textit{incoming} and \textit{outgoing }links
. A node without any incoming link is a \textit{source} node of the
network. Packets are transmitted from the source to a set of destination
nodes, i.e. \textit{receivers}, over error-free links, which is still
applicable to present-day wireline networks\footnote{Wireless network coding was introduced in \cite{Katti:2008}.}.
\begin{figure}[H]
\caption{Incoming links and outgoing links of a node in network coding \label{fig:incoming_links}}

\centering{}\includegraphics[height=0.15\paperwidth]{./figures/incoming_edges}
\end{figure}
In simple routing, information is transmitted from the source to receivers
through a chain of \textit{intermediate nodes} by a method known as
store-and-forward \cite{Yeung:2006}. In this method, the information
transmission can be represented as \textit{packet transmission}s on
links and the packets received from an incoming link of an intermediate
node, i.e. \textit{incoming packets}, can only be forwarded to a next
node via an outgoing link as its \textit{outgoing packets}. \textit{Network
coding} was though first introduced in Ahlswede et al.'s seminal paper
\cite{Ahlswede:2000} as ``coding at a node in a network'', where
coding means an arbitrary combination of received packets on a node's
incoming links for an outgoing packet. It means that each intermediate
node in the network (not only at the source) is allowed to forward
a \textit{function} of their incoming packets, e.g. $y_{1}=f_{1}(o_{1},o_{2})$
in Figure \ref{fig:incoming_links}. These functions are not required
to be injective functions, i.e. some outgoing packets can be duplicated
from a map of same incoming packets, which motivates our study in
Section \ref{sec:Network_e1l1h3rs4}. A \textit{network code} is a
set of these functions of the packets on the links of the network
\cite{Wachter-Zeh:2018}. A network code is called a \textit{solution}
for the network or the network is \textit{solvable}, if there exists
an assignment of all the functions on all the links of the network
such that each receiver can recover its requested packets from its
incoming packets. If these functions are linear, we obtain a \textit{linear
network coding solution}, and we do not consider \textit{nonlinear
solution} throughout this thesis. Each function on a link consists
of \textit{coding coefficients} for each incoming packet. The coding
coefficients form a coefficient vector whose length is equal to the
number of each node's incoming links, and this coefficient vector
is called the \textit{local coding vector}, which is distinguished
with \textit{global coding vector} defined in Section \ref{subsec:Matrix-channel}.
If the coding coefficients and the packets are scalars, a solution
of linear network coding is called \textit{scalar solution}. Based
on these coding coefficients, K\"otter and M\'edard provided in
\cite{Koetter:2003} an algebraic formulation for the linear network
coding problem and its scalar solvability.

\section{Advantages of Network Coding \label{sec:Advantages-of-NC}}

\paragraph{Throughput gain and reduced complexity}

Network coding gives a potential gain in throughput by communicating
more information with fewer packet transmissions compared to the routing
method. The butterfly network in \cite{Ahlswede:2000} as a multicast
in a wireline network is a standard example for an increase of throughput.
\begin{figure}[H]
\caption{The butterfly network \label{fig:The-butterfly-network}}

\centering{}\includegraphics[width=0.5\paperwidth]{./figures/ahlswede_butterfly_network}
\end{figure}
In Figure \ref{fig:The-butterfly-network}, we denoted a receiver
by ``Recv'', which is used for all of figures in this study. With
the help of network coding, both Recv 1 and Recv 2 can recover $x_{1}$
and $x_{2}$ by a bitwise XOR in Figure \ref{fig:The-butterfly-network}(c).
Without network coding, an additional transmission between Node 3
and 4 must be supplemented to communicate the contents of 2 packets
$x_{1}$ and $x_{2}$ from the source to Recv 1 and Recv 2, i.e. we
must communicate $x_{1}$ or $x_{2}$ separately on this link twice
under routing in Figure \ref{fig:The-butterfly-network}(a) and (b). 

\paragraph{Robustness and security}

\textit{Packet loss} is a particular issue in wireless packet networks
due to several reasons, e.g. buffer overflow or communication failures
\cite{Ho:2008}. Sharing a common concept with Erasure Coding (EC)
by exploiting a degree of redundancy to packets on any vertices in
the network, the receivers are able to successfully recover the original
packets from a large number of packet losses, e.g. $101\circledast10\circledast1$.
The only difference is that packets are only encoded by the source
in EC \cite{Fujimura:2008}. This problem is dealed by acknowledgement
messages in the mechanism of transmission control protocol (TCP) \cite{Ho:2008}.
Network coding offers both benefits and drawbacks regarding to security.
For example, node 4 is operated by an eavesdropper and it obtains
only the packet $x_{1}\oplus x_{2}$, so it cannot obtain either $x_{1}$
or $x_{2}$ and the communication is secure. Alternatively, if the
eavesdropper controls node 3, it can anonymously send a fake packet
masquerading as $x_{1}\oplus x_{2}$, which is difficult to detect
in network coding \cite{Ho:2008}.

\paragraph{From scalar network coding to vector network coding}

Ebrahimi and Fragouli \cite{Ebrahimi:2011} have extended the algebraic
approach in \cite{Koetter:2003} to \textit{vector network coding}.
Here, all packets are vectors of length $t$, and the coding coefficients
are $\left[t\times t\right]$ matrices. The network code is therefore
a set of functions consisting of $\left[t\times t\right]$ coding
matrices, and is called \textit{vector solution} if all receivers
can recover their requested information for such coding marices. The
motivation of vector netwok coding is that there exists networks do
not have scalar solutions, but are solvable by vector routing, e.g
\cite{Medard:2003}. Although it was shown that not every solvable
network has a vector solution in \cite[Lemma II.2]{Dougherty:2005},
Das and Rai proved in \cite{Das:2016} that there exists a network
with a vector solution of dimension $m$ but with no vector solution
over any finite field whose the dimension is less than $m$. When
we refer the \textit{alphabet size} of a network coding solution,
we mean the field size $q_{\mathrm{s}}$ or $q_{v}$ of the finite
field $\ensuremath{\mathbb{F}}_{q_{\mathrm{s}}}$ or $\ensuremath{\mathbb{F}}_{q_{v}}$
respectively for such a scalar solution or a vector solution. The
alphabet size is an important parameter determining the amount of
computation performed at each network node \cite{Wachter-Zeh:2018}.
The problem of finding the minimum required alphabet size of a (linear
or nonlinear) scalar network code for a certain multicast network
is NP-complete \cite{Langberg:2009,Lehman:2004,Gone:2018}. This thesis
focuses on determining the sovability of networks to measure the gap,
and our considered networks in Section \ref{sec:Description_GCN}
consist only error-free links, we therefore do not consider error
correction here. Furthermore, we consider the solvability of networks
by proving an existence of an assignment for all functions such that
all receiver can recover its requested information, so the functions
or coding coefficients are clearly chosen instead of being arbitrary
or random as the ones used in \cite{Ho:2003,Ahlswede:2000}. We later
distinguish scalar and vector network coding more specifically in
Section \ref{subsec:Matrix-channel} and \ref{sec:Network-Coding-for-GCN}.

\section{Network as a matrix channel \label{subsec:Matrix-channel}}

To formulate a network coding problem, the source has a set of disjoint
messages referred to packets on links which are either symbols from
$\ensuremath{\mathbb{F}}_{q_{\mathrm{s}}}$ (scalar coding) or vectors
of length $t$ over $\ensuremath{\mathbb{F}}_{q}$ (vector coding)
as mentioned in Section \ref{sec:What-is-NC}. Each receiver $R_{j},j\in\left\{ 1,\ldots,N\right\} $
requests a subset of same $h$ messages from the source. Through all
the functions on the links from the source to each receiver, the receiver
obtains several linear combinations of the $h$ messages to form a
linear system of equations for its requested messages. The coefficients
of a linear combination is called \textit{global coding vector} \cite{Sanders:2003}.
The linear equation system that any receiver $R_{j}$ has to solve
is as following:
\begin{equation}
\begin{array}{c|c}
\mathrm{Scalar} & \mathrm{Vector}\\
\underset{\ensuremath{\mathbb{F}}_{q_{\mathrm{s}}}^{s}}{\underbrace{\left[\begin{array}{c}
y_{j}^{\left(1\right)}\\
\vdots\\
y_{j}^{\left(s\right)}
\end{array}\right]}}=\underset{\ensuremath{\mathbb{F}}_{q_{\mathrm{s}}}^{s\times h}}{\underbrace{\boldsymbol{A}_{j}}}\cdot\underset{\ensuremath{\mathbb{F}}_{q_{s}}^{h}}{\underbrace{\left[\begin{array}{c}
x_{1}\\
\vdots\\
x_{h}
\end{array}\right]}} & \underset{\ensuremath{\mathbb{F}}_{q}^{st}}{\underbrace{\left[\begin{array}{c}
\boldsymbol{y}_{j}^{\left(1\right)}\\
\vdots\\
\boldsymbol{y}_{j}^{\left(s\right)}
\end{array}\right]}}=\underset{\ensuremath{\mathbb{F}}_{q}^{st\times th}}{\underbrace{\boldsymbol{A}_{j}}}\cdot\underset{\ensuremath{\mathbb{F}}_{q}^{th}}{\underbrace{\left[\begin{array}{c}
\boldsymbol{x}_{1}\\
\vdots\\
\boldsymbol{x}_{h}
\end{array}\right]}}
\end{array}\label{eq:linear_system}
\end{equation}
The transfer matrix $\boldsymbol{A}_{j}$ contains the links' global
coding vectors, which are combined by the coefficients of linear combinations
on $\alpha l$ links from $\alpha$ nodes and $\epsilon$ direct-links
to the corresponding receiver $R_{j}$:
\[
\begin{array}{c|c}
\mathrm{Scalar} & \mathrm{Vector}\\
\boldsymbol{A}_{j}=\left[\begin{array}{c}
\boldsymbol{a}^{\left(r_{1}\right)}\\
\vdots\\
\boldsymbol{a}^{\left(r_{\alpha l}\right)}\\
\boldsymbol{b}^{\left(\epsilon\left(j-1\right)+1\right)}\\
\vdots\\
\boldsymbol{b}^{\left(\epsilon j\right)}
\end{array}\right] & \boldsymbol{A}_{j}=\left[\begin{array}{c}
\boldsymbol{A}^{\left(r_{1}\right)}\\
\vdots\\
\boldsymbol{A}^{\left(r_{\alpha l}\right)}\\
\boldsymbol{B}^{\left(\epsilon\left(j-1\right)+1\right)}\\
\vdots\\
\boldsymbol{B}^{\left(\epsilon j\right)}
\end{array}\right]
\end{array}
\]
In general, the network is represented as a matrix channel for both
scalar and vector coding: $\boldsymbol{Y}_{j}=\boldsymbol{A}_{j}\cdot\boldsymbol{X}$.
In our study, the network structure is known, i.e. we reconstruct
$\boldsymbol{X}$ with knowing $\boldsymbol{A}_{j}$, so our network
is coherent. A network is \textit{sovable} or a network code is a
\textit{solution}, if each receiver can reconstruct its requested
messages or solve the system with a unique solution for scalars $x_{1},\ldots,x_{h}$,
or vectors $\boldsymbol{x}_{1},\ldots,\boldsymbol{x}_{h}$. In network
coding problems, we want to find global coding vectors such that the
matrix $\boldsymbol{A}_{j}$ has full-rank for every $j=1,\ldots,N$,
and such that $q_{\mathrm{s}}$ or $q^{t}$ is minimized as mentioned
in Section \ref{sec:Advantages-of-NC}. In Example \ref{ex:scalar_vector_mapping},
we provide a vector solution of field size $q$ and dimension $t$,
which has the same alphabet size as a scalar solution of field size
$q^{t}$.

To summarize the notations of both scalar and vector coding, we represent
them as in Table~\ref{tab:notations}:
\begin{table}[H]
\caption{Notations of network coding}

\label{tab:notations} 
\centering{}%
\begin{tabular}{|>{\centering}p{0.2\paperwidth}|c|c|}
\hline 
 & Scalar Coding & Vector coding\tabularnewline
\hline 
\hline 
Source Messages/Packets & $\begin{array}{c}
x_{1},\ldots,x_{h}\in\ensuremath{\mathbb{F}}_{q_{\mathrm{s}}}\\
\boldsymbol{x}\in\ensuremath{\mathbb{F}}_{q_{\mathrm{s}}}^{h}
\end{array}$ & $\begin{array}{c}
\boldsymbol{x}_{1},\ldots,\boldsymbol{x}_{h}\in\ensuremath{\mathbb{F}}_{q}^{t}\\
\boldsymbol{x}\in\ensuremath{\mathbb{F}}_{q}^{th}
\end{array}$\tabularnewline
\hline 
Global Coding Vectors Of Receiver $R_{j}$ & $\begin{array}{c}
\boldsymbol{a}^{\left(r_{1}\right)},\ldots,\boldsymbol{a}^{\left(r_{\alpha l}\right)}\in\ensuremath{\mathbb{F}}_{q_{\mathrm{s}}}^{h}\\
\boldsymbol{b}^{\left(\epsilon\left(j-1\right)+1\right)},\ldots,\boldsymbol{b}^{\left(\epsilon j\right)}\in\ensuremath{\mathbb{F}}_{q_{\mathrm{s}}}^{h}
\end{array}$ & $\begin{array}{c}
\boldsymbol{A}^{\left(r_{1}\right)},\ldots,\boldsymbol{A}^{\left(r_{\alpha l}\right)}\in\ensuremath{\mathbb{F}}_{q}^{t\times th}\\
\boldsymbol{B}^{\left(\epsilon\left(j-1\right)+1\right)},\ldots,\boldsymbol{B}^{\left(\epsilon j\right)}\in\ensuremath{\mathbb{F}}_{q}^{t\times th}
\end{array}$\tabularnewline
\hline 
Transfer Matrix Of Receiver $R_{j}$ & $\boldsymbol{A}_{j}\in\ensuremath{\mathbb{F}}_{q_{\mathrm{s}}}^{s\times h}$ & $\boldsymbol{A}_{j}\in\ensuremath{\mathbb{F}}_{q}^{st\times th}$\tabularnewline
\hline 
Packets On Receiver $R_{j}$ & $\begin{array}{c}
y_{j}^{\left(1\right)},\ldots,y_{j}^{\left(s\right)}\in\ensuremath{\mathbb{F}}_{q_{\mathrm{s}}}\\
\boldsymbol{y}_{j}\in\ensuremath{\mathbb{F}}_{q_{\mathrm{s}}}^{s}
\end{array}$ & $\begin{array}{c}
\boldsymbol{y}_{j}^{\left(1\right)},\ldots,\boldsymbol{y}_{j}^{\left(s\right)}\in\ensuremath{\mathbb{F}}_{q}^{t}\\
\boldsymbol{Y}_{j}\in\ensuremath{\mathbb{F}}_{q}^{st}
\end{array}$\tabularnewline
\hline 
Number of nodes & $r_{\mathrm{scalar}}$ & $r_{\mathrm{vector}}$\tabularnewline
\hline 
\end{tabular}
\end{table}

\begin{rem}
By using the vector coding, the upper bound number of solutions increases
from $q^{tkh}$ to $q^{t^{2}kh}$. Therefore, vector network coding
offers more freedom in choosing the coding coefficients than does
scalar linear coding for equivalent alphabet sizes, and a smaller
alphabet size might be achievable \cite{Ebrahimi:2011}. By this advantage,
we can have higher number of receivers, i.e. higher number of nodes,
in vector network coding.
\end{rem}

\section{Network Model}

\subsection{Multicast Networks as Generalized Combination Networks}

A class of networks which is mainly studied is the class of multicast
networks. It can be one-to-many or many-to-many distribution \cite{Harte:2008}.
In this study, we target one-to-many multicast network with the distribution
of a data packet to a group of users \cite{Zhang:2012}. An interesting
network structure often used for multicast networks in network coding
is called Combination Network (CN) and denoted by $\mathcal{N}_{h,r,s}$.
Many examples in previous studies demonstrating the advantage of network
coding have used structures identical or similar to that of CN. We
mention a few examples to emphasize CN's importance in the study of
network coding. In Figure \ref{fig:butterfly_nw_cn}, the butterfly
network that is often used as a first example to motivate network
coding, e.g. \cite[Fig. 7]{Ahlswede:2000} and \cite[Fig. 1]{Sanders:2003},
is isomorphic to $\mathcal{N}_{h,r=3,s=2}$, if we consider it as
an undirected network \cite{Maheshwar:2012}. The $\mathcal{N}_{h,r=3,s=2}$
itself was also used in the first study of network coding \cite{Ahlswede:2000}.
Other CNs, i.e. $\mathcal{N}_{h,r=4,s=2}$ and $\mathcal{N}_{h,r=6,s=3}$,
were also used as examples to demonstrate the advantage of network
coding in \cite[Fig. 2]{Sanders:2003} and \cite[Fig. 2]{Jaggi:2005}
respectively. The general structure of CN was also introduced and
discussed in \cite[Sec. 4.3]{Fragouli:2006}, \cite[Sec. 4.1]{Yeung:2006},
\cite{Ngai:2004,Xiao:2007}.
\begin{figure}[H]
\caption{The butterfly network is represented as a combination network \label{fig:butterfly_nw_cn}}

\centering{}\includegraphics[width=0.5\paperwidth]{./figures/ahlswede_butterfly_network_CN}
\end{figure}
A generalization of a CN \cite{Riis:2006} is called generalized combination
network (GCN). GCN defined in \cite{Etzion:2016,Wachter-Zeh:2018}
was used to prove that vector network coding outperforms scalar linear
network coding , in multicast networks, with respect to the \textit{alphabet
size}, using rank-metric codes and Grassmannian codes. A comparison
between the required alphabet size for a scalar linear solution, a
vector solution, and a scalar nonlinear solution, of the same multicast
network is an important problem. Etzion and Wachter-Zeh introduced
a \textit{gap} in \cite{Etzion:2016} as the difference between the
smallest alphabet size for which a scalar linear solution exists and
the smallest alphabet size for which they can construct a vector solution.
They have found bounds on the gap for several network families of
GCN in \cite{Etzion:2016,Wachter-Zeh:2018}, but no gap for the GCN
networks with 3 messages has been found, i.e. $(1,1)-\mathcal{N}_{3,r,4}$,
where we denote GCN by $(\epsilon,l)-\mathcal{N}_{h,r,s}$. Therefore,
a combinatorial approach is first introduced in this thesis to prove
an existence of a vector solution outperforming the optimal scalar
linear solution with $q^{t^{2}/4+\mathcal{O}(t)}$. We then further
extend the approach for a family of GCN called One-Direct Link Combination
Network, i.e. $(1,1)-\mathcal{N}_{h,r,s}$. More formal definitions
of the gap and GCN can be found in Section \ref{sec:Description_GCN}.

\subsection{Comparison between scalar and vector solutions by the gap size \label{subsec:Comparison-between-scalar-and-vector-sol}}

A \textit{gap} represents the difference between the smallest field
(alphabet) size for which a scalar linear solution exists and the
smallest alphabet size for which we can construct a vector solution.
In this study, we define a solvable vector network coding over the
field size $\ensuremath{\mathbb{F}}_{q}^{t}$, and we find the lower
bound of the maximum number of nodes such vector solution can achieve,
i.e. $r_{\mathrm{max,vector}}=f_{\mathrm{1}}(q,t,\alpha,h)\geq f_{\mathrm{1,lower\,bound}}\left(q,t,\alpha,h\right)$,
with $f_{\mathrm{1}}:\mathbb{Z}\rightarrow\mathbb{Z}$. Meanwhile,
we has a scalar solution for the same network existing if and only
if $r\leq r_{\mathrm{max,scalar}}=f_{\mathrm{2}}\left(q_{\mathrm{s}}\right)\leq f_{\mathrm{2,upper\,bound}}\left(q_{\mathrm{s}}\right)$,
with $f_{\mathrm{2}}:\mathbb{Z}\rightarrow\mathbb{Z}$. To find the
field size $q_{\mathrm{s}}$ required for a scalar solution to reach
the maximum achievable vector solution's nodes in this setting, we
consider $r_{\mathrm{max,scalar}}=f_{\mathrm{2}}\left(q_{\mathrm{s}}\right)=f_{\mathrm{1}}(q,t,\alpha,h)=r_{\mathrm{max,vector}}$
and express $q_{\mathrm{s}}$ in term of $q,t,\alpha,h$. In other
words, we compute such field size as following, $q_{\mathrm{s}}\left(q,t,\alpha,h\right)=\mathrm{\mathrm{min}\left\{ q_{\mathrm{s}}:f_{\mathrm{2}}\left(q_{\mathrm{s}}\right)\geq f_{\mathrm{1}}\left(q,t,\alpha,h\right)\right\} }$,
we then calculate the gap by $g=q_{\mathrm{s}}-q_{v}=q_{\mathrm{s}}-q^{t}$,
which is similar to the way to compute the gap in \cite{Wachter-Zeh:2018}.
In this thesis, we focus on a lower bound of the gap as following,
\[
\text{g}\geq g_{\mathrm{lower\,bound}}=q_{\mathrm{s,min,from\,bound}}(q,t,\alpha,h)-q^{t}
\]

, where $q_{\mathrm{s,min,from\,bound}}=\mathrm{min}\left\{ q_{\mathrm{s}}:f_{\mathrm{2,upper\,bound}}\left(q_{\mathrm{s}}\right)\geq f_{\mathrm{1,lower\,bound}}\left(q,t,\alpha,h\right)\right\} $. 

Throughout this study, we show that vectors solutions significantly
reduce the required alphabet size by the lower bound of the gap denoted
by $g_{\mathrm{lower\,bound}}$.

\clearpage
	%%%%%%%%%%%%%%%%%%%%%%%%%%%%%%%%%%%%%%%%%%%%%%%
\chapter{Generalized Combination Network} \label{chap:general_network}
%%%%%%%%%%%%%%%%%%%%%%%%%%%%%%%%%%%%%%%%%%%%%%%



\clearpage
    %%%%%%%%%%%%%%%%%%%%%%%%%%%%%%%%%%%%%%%%%%%%%%%
\chapter{Combinatorial Results} \label{chap:comb_res}
%%%%%%%%%%%%%%%%%%%%%%%%%%%%%%%%%%%%%%%%%%%%%%%

In previous studies \cite{Wachter-Zeh:2018}, no general vector solution
outperforming scalar network coding was found for multicast networks
with $h=3$ messages. Hence, we start with a probabilistic argument
to prove that there exists a vector solution outperforming the optimal
linear solution for the $\left(\epsilon=1,\ell=1\right)-\mathcal{N}_{h=3,r,s=4}$
network. Then we generalize the proof to the $\left(\epsilon=1,\ell=1\right)-\mathcal{N}_{h,r,s}$
network and the $\left(\epsilon=1,\ell>1\right)-\mathcal{N}_{h=2\ell,r,s=2\ell+1}$
network. As explained in Section \ref{sec:What-is-NC}, multiple parallel
links $\ell$ of a data unit help us to show networks with large-capacity
transmission between source and receivers. The direct links among
them are not really usual in reality, i.e. a server and a client often
has long-distance connection thorugh multiple intermediate nodes,
it is thus interesting to study networks with $\epsilon=1$. We formally
use $r_{scalar}$ and $r_{vector}$ to distinguish the $r$ parameter
of GCN for scalar solutions and vector solutions to compare their
gap. Because they both have the same meaning as a number of intermediate
nodes in a network, we use $r$ when we need to state a vector solution
or a scalar solution exists under some conditions of $r$.

\section{$\left(\epsilon=1,\ell=1\right)-\mathcal{N}_{h=3,r,s=4}$ Network
\label{sec:Network_e1l1h3rs4}}

\begin{figure}[H]
\caption{The $(\epsilon=1,\ell=1)-\mathcal{N}_{h=3,r,s=4}$ network\label{fig:nw_e1_l1_h3_r_s4}}

\centering{}\includegraphics[width=0.5\paperwidth]{E:/Documents/TUM/THESIS/thesisCOD_Ha/figures/nw_e1_l1_h3_r_s4}
\end{figure}
In this subsection, we derive a lower bound on the maximum number
of receivers for the $\left(\epsilon=1,\ell=1\right)-\mathcal{N}_{h=3,r,s=4}$
network. Due to $\alpha=3$, the number of receivers is $N=\left(\begin{array}{c}
r_{vector}\\
3
\end{array}\right)$ by definition in Section \ref{sec:Description_GCN}. To derive the
lower bound, we introduce a rank requirement on incoming packets to
each receiver. 
\begin{figure}[H]
\caption{The vector network coding of $(\epsilon=1,l=1)-\mathcal{N}_{h=3,r,s=4}$
represents as a matrix problem\label{fig:rk_h3}}

\centering{}\includegraphics[width=0.6\paperwidth]{E:/Documents/TUM/THESIS/thesisCOD_Ha/figures/rk_h3}
\end{figure}
Following to Equation \ref{eq:linear_system}, each receiver $R_{j}$
must solve a linear equation system of 3 variables with 4 equations
to recover $h=3$ messages as below:
\[
\left[\begin{array}{c}
\boldsymbol{y}_{j}^{\left(r_{1}\right)}\\
\boldsymbol{y}_{j}^{\left(r_{2}\right)}\\
\boldsymbol{y}_{j}^{\left(r_{3}\right)}\\
\boldsymbol{y}_{j}^{\left(r_{4}\right)}
\end{array}\right]=\boldsymbol{A}_{j}\cdot\underline{x}=\left[\begin{array}{c}
\boldsymbol{A}_{j}^{\left(r_{1}\right)}\\
\boldsymbol{A}_{j}^{\left(r_{2}\right)}\\
\boldsymbol{A}_{j}^{\left(r_{3}\right)}\\
\boldsymbol{A}_{j}^{\left(r_{4}\right)}
\end{array}\right]\cdot\left[\begin{array}{c}
\boldsymbol{x}_{1}\\
\boldsymbol{x}_{2}\\
\boldsymbol{x}_{3}
\end{array}\right],
\]
with $\boldsymbol{x}_{i},\boldsymbol{y}_{j}^{\left(v\right)}\in\ensuremath{\mathbb{F}}_{q}^{t},\boldsymbol{A}_{j}^{\left(r_{v}\right)}\in\ensuremath{\mathbb{F}}_{q}^{t\times3t}$
for $v=1,\ldots,4$, and $\boldsymbol{A}_{j}^{\left(r_{1}\right)},\ldots,\boldsymbol{A}_{j}^{\left(r_{3}\right)}$
must be distinct.

The network is solvable, if $\boldsymbol{A}_{j}$ has full-rank, i.e.
$\boldsymbol{A}_{j_{v}}$ must satisfy:
\[
rk\left[\begin{array}{c}
\boldsymbol{A}_{j}^{\left(r_{1}\right)}\\
\boldsymbol{A}_{j}^{\left(r_{2}\right)}\\
\boldsymbol{A}_{j}^{\left(r_{3}\right)}\\
\boldsymbol{A}_{j}^{\left(r_{4}\right)}
\end{array}\right]\geq3t
\]
Because coding coefficients for $\boldsymbol{A}_{j}^{\left(r_{4}\right)}$
can be independently chosen for any receiver $R_{j}$, there always
exists $\boldsymbol{A}_{j}^{\left(r_{4}\right)}$ such that,
\begin{equation}
rk\left[\begin{array}{c}
\boldsymbol{A}_{j}^{\left(r_{1}\right)}\\
\boldsymbol{A}_{j}^{\left(r_{2}\right)}\\
\boldsymbol{A}_{j}^{\left(r_{3}\right)}
\end{array}\right]\geq2t\label{eq:rk_rqm_e1l1h3s4}
\end{equation}
if and only if $rk\left[\boldsymbol{A}_{j}\right]\geq3t$.

By this constraint, the problem is thus described as below:
\[
\underset{rk\left[\boldsymbol{A}_{j}\right]\geq3t}{min}\,r_{vector}
\]
$\boldsymbol{A}_{j}^{\left(r_{1}\right)},\boldsymbol{A}_{j}^{\left(r_{2}\right)},\boldsymbol{A}_{j}^{\left(r_{3}\right)}$
are matrices formed on any 3 of $r$ links from nodes to receivers,
i.e. these 3 matrices are randomly chosen from a set of $r$ matrices.
We formalize the problem by an approach with Lov\'asz local lemma,
which was initially proposed by Schwartz in \cite{MosheSchwartz:2018}.
\begin{lem}[Symmetric Lov\'asz local lemma (LLL) \cite{Schwarz:2013}]
 A set of events $\mathcal{E}_{i}$, with $i=1,\ldots,n$, such that
each event occurs with probability at most $p$. If each event is
independent of all others except for at most $d$ of them and $4dp\leq1$,
then: $Pr\left[\stackrel[i=1]{n}{\bigcap}\overline{\mathcal{E}}_{i}\right]>0$.
\label{thm:LLL}
\end{lem}
%
\begin{lem}
\label{lem:prob_p_LLL_formula} Let $Pr\left[\mathcal{E}_{i}\right]=Pr\left[rk\left[\begin{array}{c}
\boldsymbol{A}_{j}^{\left(r_{1}\right)}\\
\boldsymbol{A}_{j}^{\left(r_{1}\right)}\\
\boldsymbol{A}_{j}^{\left(r_{3}\right)}
\end{array}\right]<2t\right]\leq p,\forall1\leq r_{1}<r_{2}<r_{3}\leq r$, and $\boldsymbol{A}_{j}^{\left(r_{1}\right)},\ldots,\boldsymbol{A}_{j}^{\left(r_{3}\right)}\in\ensuremath{\mathbb{F}}_{q}^{t\times3t}$,
then, 
\[
p\leq\Theta\left(q^{-t^{2}-2t-1}\right),\forall t\geq2.
\]
\end{lem}
\begin{proof}
In Lemma \ref{thm:LLL}, each event $\mathcal{E}_{i}$ is a \textit{bad
event}, whose occurrence is undesirable. Following to Equation \ref{eq:rk_rqm_e1l1h3s4},
such a event occurs when $rk\left[\boldsymbol{A}_{j}\right]<3t$,
and its probability is bounded by $p$ (with $0\leq p\leq1$) as following,
\begin{equation}
Pr\left[\mathcal{E}_{i}\right]=Pr\left[rk\left[\begin{array}{c}
\boldsymbol{A}_{j}^{\left(r_{1}\right)}\\
\boldsymbol{A}_{j}^{\left(r_{1}\right)}\\
\boldsymbol{A}_{j}^{\left(r_{3}\right)}
\end{array}\right]<2t\right]\leq p.\label{eq:p_in_LLL}
\end{equation}
Regarding to the left-hand side,
\begin{eqnarray}
Pr\left[rk\left[\begin{array}{c}
\boldsymbol{A}_{j}^{\left(r_{1}\right)}\\
\boldsymbol{A}_{j}^{\left(r_{2}\right)}\\
\boldsymbol{A}_{j}^{\left(r_{3}\right)}
\end{array}\right]<2t\right] & = & \stackrel[i=0]{2t-1}{\mathop{\sum}}Pr\left[rk\left[\begin{array}{c}
\boldsymbol{A}_{j}^{\left(r_{1}\right)}\\
\boldsymbol{A}_{j}^{\left(r_{2}\right)}\\
\boldsymbol{A}_{j}^{\left(r_{3}\right)}
\end{array}\right]=i\right]\nonumber \\
 & \overset{1}{=} & \stackrel[i=0]{2t-1}{\mathop{\sum}}\frac{N_{i,m,n}}{q^{m\cdot n}}\nonumber \\
 & = & \stackrel[i=0]{2t-1}{\mathop{\sum}}\frac{\stackrel[j=0]{i-1}{\mathop{\prod}}\frac{\left(q^{m}-q^{j}\right)\left(q^{n}-q^{j}\right)}{q^{i}-q^{j}}}{q^{m\cdot n}}\nonumber \\
 & \overset{2}{=} & \stackrel[i=0]{2t-1}{\mathop{\sum}}\frac{\stackrel[j=0]{i-1}{\mathop{\prod}}\frac{\left(q^{3t}-q^{j}\right)^{2}}{q^{i}-q^{j}}}{q^{9t^{2}}}.\label{eq:p_eq_h3}
\end{eqnarray}

(1): The formula for the umber of $\left[m\times n\right]$ matrices
of rank $i$ over $\ensuremath{\mathbb{F}}_{q}$ was proved in \cite{Overbeck:2007}.

(2): $\boldsymbol{A}_{j}^{\left(r_{1}\right)},\boldsymbol{A}_{j}^{\left(r_{2}\right)},\boldsymbol{A}_{j}^{\left(r_{3}\right)}$
vertically together form a $\left[3t\times3t\right]$ matrix.

We consider the numerator of Equation (\ref{eq:p_eq_h3}): $\stackrel[j=0]{i-1}{\mathop{\prod}}\frac{\left(q^{3t}-q^{j}\right)^{2}}{q^{i}-q^{j}}=\frac{p_{N}^{(i)}(q)}{p_{D}^{(i)}(q)}=p^{(i)}(q)$.

Due to $i$-times product and large $t$: $\left.\begin{array}{c}
deg\left(p_{N}^{(i)}(q)\right)=q^{i6t}\\
deg\left(p_{D}^{(i)}(q)\right)=q^{i^{2}}
\end{array}\right\} \Rightarrow p^{(i)}(q)\approx q^{i6t-i^{2}}$.

Therefore, we have: $\stackrel[i=0]{2t-1}{\mathop{\sum}}\stackrel[j=0]{i-1}{\mathop{\prod}}\frac{\left(q^{3t}-q^{j}\right)^{2}}{q^{i}-q^{j}}=\stackrel[i=0]{2t-1}{\mathop{\sum}}p^{(i)}(q)\approx\stackrel[i=0]{2t-1}{\mathop{\sum}}q^{i6t-i^{2}}$.

To maximize the sum, we set derivation of it to 0 and find the corresponding
root: 
\begin{eqnarray*}
 & \left(i6t-i^{2}\right)^{'} & =0\\
\Leftrightarrow & 6t-2i & =0\\
\Leftrightarrow & i & =3t
\end{eqnarray*}
However, the upper limit of the sum is $\left(2t-1\right)$, which
is less than $3t$ for all $t\geq2$.
\[
\Rightarrow max\left\{ q^{i6t-i^{2}}:i=0,2\ldots,2t-1\right\} =\left.q^{i6t-i^{2}}\right|_{i=2t-1}=q^{8t^{2}-2t-1}
\]
Hence, by using the exact bound $\Theta$, we have:
\[
\underset{i}{max}\left[\stackrel[i=0]{2t-1}{\mathop{\sum}}p^{(i)}(q)\right]\in\Theta\left(max\left\{ q^{i6t-i^{2}}:i=1,2\ldots,2t-1\right\} \right)=\Theta\left(q^{8t^{2}-2t-1}\right)
\]
\[
\Rightarrow\underset{i}{max}\left[\frac{\stackrel[i=0]{2t-1}{\mathop{\sum}}p^{(i)}(q)}{q^{9t^{2}}}\right]\in\Theta\left(q^{-t^{2}-2t-1}\right)
\]
\[
\Rightarrow p\leq\Theta\left(q^{-t^{2}-2t-1}\right)
\]
\end{proof}
\begin{lem}
Let $Pr\left[\mathcal{E}_{i}\right]=Pr\left[rk\left[\begin{array}{c}
\boldsymbol{A}_{j}^{\left(r_{1}\right)}\\
\boldsymbol{A}_{j}^{\left(r_{1}\right)}\\
\boldsymbol{A}_{j}^{\left(r_{3}\right)}
\end{array}\right]<2t\right]\leq p,\forall1\leq r_{1}<r_{2}<r_{3}\leq r$ with $\boldsymbol{A}_{j}^{\left(r_{1}\right)},\ldots,\boldsymbol{A}_{j}^{\left(r_{3}\right)}\in\ensuremath{\mathbb{F}}_{q}^{t\times3t}$,
and each event $\mathcal{E}_{i}$ is independent of all others except
for at most $d$ of them, then $d\leq\frac{3}{2}r^{2}$. \label{lem:dependecy_d_LLL}
\end{lem}
\begin{proof}
Because $d$ is a function of $r$ in our problem, we denote $d$
in Lemma \ref{thm:LLL} specifically by $d(r)$. The Local lemma \ref{thm:LLL}
allows dependence among at most $d(r)$ events, i.e. each event considered
under the Local lemma must no be independent to each other.

Furthermore, $\boldsymbol{A}_{j}^{\left(r_{1}\right)},\boldsymbol{A}_{j}^{\left(r_{2}\right)},\boldsymbol{A}_{j}^{\left(r_{3}\right)}$
are matrices formed on any 3 of $r$ links from nodes to receivers.
Assume that 1 of $r$ links is firstly chosen to form the matrix $\boldsymbol{A}_{j}^{\left(r_{1}\right)}$,
then there are $\left(\begin{array}{c}
r-1\\
2
\end{array}\right)$ posibilities left to choose $\boldsymbol{A}_{j}^{\left(r_{2}\right)}$
and $\boldsymbol{A}_{j}^{\left(r_{3}\right)}$ from $r-1$ links.
However, such a link can also be chosen firstly to form $\boldsymbol{A}_{j}^{\left(r_{2}\right)}$
or $\boldsymbol{A}_{j}^{\left(r_{3}\right)}$ instead. Therefore,
we have a upper bound for $d(r)$ as following,
\[
d(r)\leq3\cdot\left(\begin{array}{c}
r-1\\
2
\end{array}\right)=3\cdot\frac{\left(r-1\right)\left(r-2\right)}{2}=\frac{3}{2}\left(r^{2}-3r+2\right)
\]
\[
\Rightarrow d(r)\leq\frac{3}{2}r^{2}
\]
\end{proof}
\begin{example}
$r\in\left\{ 1,2,3,4\right\} $. Our sample space to draw any 3 matrices
with orders has in total $\left(\begin{array}{c}
4\\
3
\end{array}\right)=4$ samples.

Assume the link 1 is firstly chosen to form the matrix $\boldsymbol{A}_{j}^{\left(r_{1}\right)}$,
we have $\left(\begin{array}{c}
3\\
2
\end{array}\right)=3$ possibilities: $\left[\begin{array}{c}
1\\
2\\
3
\end{array}\right],\left[\begin{array}{c}
1\\
2\\
4
\end{array}\right]$ and $\left[\begin{array}{c}
1\\
3\\
4
\end{array}\right]$. The existence of $1$ in $\left[\begin{array}{c}
1\\
2\\
3
\end{array}\right]$ gives us some information on $rk\left[\begin{array}{c}
1\\
2\\
4
\end{array}\right]$ or $rk\left[\begin{array}{c}
1\\
3\\
4
\end{array}\right]$, which means these events $rk\left[\begin{array}{c}
\boldsymbol{A}_{j}^{\left(r_{1}\right)}\\
\boldsymbol{A}_{j}^{\left(r_{1}\right)}\\
\boldsymbol{A}_{j}^{\left(r_{3}\right)}
\end{array}\right]<2t$ are dependent. 

We also have 3 posibilities, when we choose the link 1 to form the
matrix $\boldsymbol{A}_{j}^{\left(r_{2}\right)}$: $\left[\begin{array}{c}
2\\
1\\
3
\end{array}\right],\left[\begin{array}{c}
2\\
1\\
4
\end{array}\right]$ and $\left[\begin{array}{c}
3\\
1\\
4
\end{array}\right]$. 

There are again 3 posibilities, when we choose the link $1$ to form
the matrix $\boldsymbol{A}_{j}^{\left(r_{3}\right)}$: $\left[\begin{array}{c}
2\\
3\\
1
\end{array}\right],\left[\begin{array}{c}
2\\
4\\
1
\end{array}\right]$ and $\left[\begin{array}{c}
3\\
4\\
1
\end{array}\right]$.

Let's denote the event $rk\left[\begin{array}{c}
1\\
2\\
3
\end{array}\right]<2t$ by $\mathcal{E}_{1}$. It is clearly to see that $Pr\left[\begin{array}{c|c}
rk\left[\begin{array}{c}
2\\
1\\
3
\end{array}\right]<2t & \mathcal{E}_{1}\end{array}\right]=1$. The bound in Lemma \ref{lem:dependecy_d_LLL} is a bit loose, because
we do not neglect unsuitable matrices, e.g. $\left[\begin{array}{c}
3\\
1\\
4
\end{array}\right]$. In the other words, there exists at most 9 events are dependent. 
\end{example}
>\textcompwordmark >\textcompwordmark > IS THIS EXAMPLE CONVINCING
ENOUGH? WHY IS IT NOT $d(r)\leq3\cdot r\cdot\left(\begin{array}{c}
r-1\\
2
\end{array}\right)=$?
\begin{thm}
If $r\leq\Omega\left(q^{t^{2}/2+\mathcal{O}\left(t\right)}\right)$,
then there exists a vector solution for the $\left(\epsilon=1,l=1\right)-\mathcal{N}_{h=3,r,s=4}$
network. \label{theo:r_for_vector_sol_e1l1h3rs4}
\end{thm}
\begin{proof}
The Local lemma \ref{thm:LLL} shows that there is a positive probability
that none of bad events occurs: $Pr\left[\stackrel[i=1]{n}{\bigcap}\overline{\mathcal{E}}_{i}\right]>0$.

By the intersection rule, none of bad events is equivalent to an event
$T$ that its set of outcomes are all desirable, i.e. these outcomes
satisfy the Equation \ref{eq:rk_rqm_e1l1h3s4}:
\[
T=\stackrel[i=1]{n}{\bigcap}\overline{\mathcal{E}}_{i}=rk\left[\begin{array}{c}
\boldsymbol{A}_{j}^{\left(r_{1}\right)}\\
\boldsymbol{A}_{j}^{\left(r_{2}\right)}\\
\boldsymbol{A}_{j}^{\left(r_{3}\right)}
\end{array}\right]\geq2t,\forall1\leq r_{1}<r_{2}<r_{3}\leq r.
\]
The probability of event $T$ indicates a measure quantifying the
likelihood that we are able to construct $rk\left[\boldsymbol{A}_{j}\right]\geq3t$
given $r$ links. It means that a vector solution exists if and only
if the Local lemma \ref{thm:LLL} is satisfied such that: $4\cdot p\cdot d(r)\leq1,\forall r\leq r_{max,vector}$.
Therefore, we must find a lower bound of $r_{max,vector}$. Furthermore,
we have $d\leq\frac{3}{2}r^{2}$ following to Lemma \ref{lem:dependecy_d_LLL},
which gives: $4\cdot p\cdot\frac{3}{2}r^{2}\leq1\Rightarrow r\leq\sqrt{\frac{1}{6p}}=r_{max,vector}$.
Thus, finding $min\left\{ r_{max,vector}\right\} $ is equivalent
to maximize $p$.

By Lemma \ref{lem:prob_p_LLL_formula}, we have $p\leq\Theta\left(q^{-t^{2}-2t-1}\right)$,
\[
\Rightarrow min\left\{ r_{max,vector}\right\} \in\Omega\left(\sqrt{\frac{1}{6p}}\right)=\Omega\left(\sqrt{\frac{1}{6q^{-t^{2}-2t-1}}}\right)=\Omega\left(q^{t^{2}/2+\mathcal{O}\left(t\right)}\right).
\]
Hence, the Local lemma in \ref{thm:LLL} is satisfied, when $r\leq\Omega\left(q^{t^{2}/2+\mathcal{O}\left(t\right)}\right)$.
None of bad events occurs, so there exists a vector solution for such
$r$.
\end{proof}
\begin{cor}
The $\left(\epsilon=1,\ell=1\right)-\mathcal{N}_{h=3,r,s=4}$ network
has a vector solution with a gap $q^{t^{2}/4+\mathcal{O}(t)}$.
\end{cor}
In \cite[Sec. VIII-C]{Wachter-Zeh:2018}, we have that $r_{max,scalar}\in\mathcal{O}\left(q_{\mathrm{s}}^{2}\right)$,
where they proved that
\begin{equation}
r_{scalar}\leq2\left[\begin{array}{c}
3\\
1
\end{array}\right]_{q_{s}}=2\left(q_{s}^{2}+q_{s}+1\right).\label{eq:r_scalar_max}
\end{equation}
Following to Section \ref{subsec:Comparison-between-scalar-and-vector-sol}
and Theorem \ref{theo:r_for_vector_sol_e1l1h3rs4}, we have the gap
size
\begin{eqnarray}
 & r_{max,scalar} & =min\left\{ r_{max,vector}\right\} \nonumber \\
\Leftrightarrow & q_{\mathrm{s}}^{2} & =q^{t^{2}/2+\mathcal{O}(t)}\nonumber \\
\Leftrightarrow & q_{\mathrm{s}} & ^{=}q^{t^{2}/4+\mathcal{O}(t)}\nonumber \\
\Rightarrow & g & =q_{\mathrm{s}}-q_{v}=q^{t^{2}/4+\mathcal{O}(t)}\label{eq:gap_e1l1h3rs4}
\end{eqnarray}
By varying $t$ in Equation (\ref{eq:p_eq_h3}), we have the following
table:
\begin{table}[H]
\caption{$r$ over variations of t\label{tab:r_over_t}}

\begin{centering}
\begin{tabular}{|c|c|c|}
\hline 
t & Scalar Solution & Vector Solution\tabularnewline
\hline 
\hline 
1 & $r_{scalar}\leq14$ & $r_{vector}\geq3$\tabularnewline
\hline 
2 & $r_{scalar}\leq42$ & $r_{vector}\geq7\,\left(67^{*},\,89^{**}\right)$\tabularnewline
\hline 
3 & $r_{scalar}\leq146$ & $r_{vector}\geq62\,\left(166^{*}\right)$ \tabularnewline
\hline 
4 & $r_{scalar}\leq546$ & $r_{vector}\geq1317$\tabularnewline
\hline 
5 & $r_{scalar}\leq2114$ & $r_{vector}\geq58472$\tabularnewline
\hline 
6 & $r_{scalar}\leq8322$ & $r_{vector}>10^{6}$\tabularnewline
\hline 
\end{tabular}
\par\end{centering}
{*}, {*}{*}: computational results in construction 1 and construction
2 respectively
\end{table}
In the table (\ref{tab:r_over_t}), the vector solution outperforms
the scalar solution when $t\geq4$ for the network $\left(\epsilon=1,\ell=1\right)-\ensuremath{N}_{h=3,r,s=4}$.
This is sufficient, we show in Section 6 computational results which
vector solutions outperform scalar solutions in case of $t=2$ and
$t=3$.

\section{$\left(\epsilon=1,\ell=1\right)-\mathcal{N}_{h,r,s}$ Network \label{sec:e1l1_nw}}

\subsection{Find the lower bound of $r_{max,vector}$}

As previous, $\boldsymbol{A}_{j}^{\left(r_{1}\right)},\ldots,\boldsymbol{A}_{j}^{\left(r_{h-\epsilon}\right)}\in\ensuremath{\mathbb{F}}_{q}^{t\times ht}$
and we need to satisfy the following:
\[
rk\left[\begin{array}{c}
\boldsymbol{A}_{j}^{\left(r_{1}\right)}\\
\vdots\\
\boldsymbol{A}_{j}^{\left(r_{h-\epsilon}\right)}
\end{array}\right]\geq ht-t\Leftrightarrow rk\left[\boldsymbol{A}_{j}\right]\geq(h-1)t
\]
We can formulate it by the following coding problem in Grassmannian:

\noindent\fbox{\begin{minipage}[t]{1\columnwidth - 2\fboxsep - 2\fboxrule}%
Find the largest set of subspaces from $\mathcal{G}_{q}\left(ht,t\right)$
such that any $\alpha$ subspaces of the set span a subspace of dimension
at least $\left(h-1\right)t$.%
\end{minipage}}Similar with $\left(\epsilon=1,\ell=1\right)-\ensuremath{N}_{3,r,4}$,
we consider $p$ to proceed the Local lemma \ref{thm:LLL}:
\begin{equation}
Pr\left[rk\left[\boldsymbol{A}_{j}\right]<(h-1)t\right]\leq p\label{eq:p_e1l1}
\end{equation}

\begin{lem}
Let $Pr\left[\mathcal{E}_{i}\right]=Pr\left[rk\left[\begin{array}{c}
\boldsymbol{A}_{j}^{\left(r_{1}\right)}\\
\vdots\\
\boldsymbol{A}_{j}^{\left(r_{h-\epsilon}\right)}
\end{array}\right]<(h-1)t\right]\leq p,\forall1\leq r_{1}<\ldots<r_{h-\epsilon}\leq r$ and $\boldsymbol{A}_{j}^{\left(r_{1}\right)},\ldots,\boldsymbol{A}_{j}^{\left(r_{h-\epsilon}\right)}\in\ensuremath{\mathbb{F}}_{q}^{t\times ht}$,
then, 
\[
p\leq\Theta\left(q^{\left(h-\alpha\right)t^{2}+\mathcal{O}(t)}\right),\forall t\geq2.
\]
\end{lem}
\begin{proof}
Regarding to the left-hand side in Equation \ref{eq:p_e1l1}:
\begin{eqnarray}
Pr\left[rk\left[\boldsymbol{A}_{j}\right]<(h-1)t\right] & = & \stackrel[i=0]{(h-1)t-1}{\mathop{\sum}}Pr\left[rk\left[\boldsymbol{A}\right]=i\right]\nonumber \\
 & \overset{1}{=} & \stackrel[i=0]{(h-1)t-1}{\mathop{\sum}}\frac{N_{i,\alpha t,ht}}{q^{\left(\alpha t\right)\left(ht\right)}}\nonumber \\
 & = & \frac{1}{q^{\left(\alpha h\right)t^{2}}}\cdot\stackrel[i=0]{(h-1)t-1}{\mathop{\sum}}\stackrel[j=0]{i-1}{\mathop{\prod}}\frac{\left(q^{\alpha t}-q^{j}\right)\left(q^{ht}-q^{j}\right)}{q^{i}-q^{j}}.\label{eq:general_nw_calc_p}
\end{eqnarray}

(1): The formula for the umber of $\left[m\times n\right]$ matrices
of rank $i$ over $\ensuremath{\mathbb{F}}_{q}$ was proved in \cite{Overbeck:2007}.

We consider firstly the product$\stackrel[j=0]{i-1}{\mathop{\prod}}\frac{\left(q^{\alpha t}-q^{j}\right)\left(q^{ht}-q^{j}\right)}{q^{i}-q^{j}}=\frac{p_{N}^{(i)}(q)}{p_{D}^{(i)}(q)}=p^{(i)}(q)$.

For $t\rightarrow\infty$: $\left.\begin{array}{c}
deg\left(p_{N}^{(i)}(q)\right)=q^{i(\alpha t+ht)}\\
deg\left(p_{D}^{(i)}(q)\right)=q^{i^{2}}
\end{array}\right\} \Rightarrow p^{(i)}(q)\approx q^{i(\alpha t+ht)-i^{2}}.$

Now, we evaluate $f(i)=i(\alpha t+ht)-i^{2}$ to find its maximum
point by its derivation : 

$\dot{f}(i^{*})=0\Leftrightarrow(\alpha t+ht)-2i^{*}=0\Leftrightarrow i^{*}=\frac{\alpha t+ht}{2}$

We then check whether this point within the range $i=0,\ldots,(h-1)t-1$
as following: $0\leq\frac{\alpha t+ht}{2}\leq(h-1)t-1$.

With regards to the lower bound: $0\leq\frac{\alpha t+ht}{2}\Leftrightarrow t\geq\frac{2}{\alpha+h}$,
which is always true due to the given $t\geq2$ and $\alpha,h\geq3$.

Regarding to the upper bound: $\frac{\alpha t+ht}{2}\leq(h-1)t-1\Leftrightarrow t\leq\frac{-2}{\alpha+2-h}$
with $\alpha+2>h$ due to the given $\alpha l+\epsilon=\alpha+1\geq h$.
This cannot happen because of $t\geq2$, i.e. this maximum point is
over then upper-range limit.
\begin{eqnarray*}
\Rightarrow & max\left\{ q^{i(\alpha t+ht)-i^{2}}:i=1,\ldots,(h-1)t-1\right\}  & =\left.q^{i(\alpha t+ht)-i^{2}}\right|_{i=(h-1)t-1}\\
 &  & =q^{\left[\left(h-1\right)\left(\alpha+1\right)\right]t^{2}-\left(\alpha-h+2\right)t-1}.
\end{eqnarray*}
Secondly, we apply the maximum value with the sum, we have:
\begin{eqnarray*}
 & max\left\{ \stackrel[i=0]{(h-1)t-1}{\mathop{\sum}}p^{(i)}(q)\right\}  & \in\Theta\left(q^{\left[\left(h-1\right)\left(\alpha+1\right)\right]t^{2}+\mathcal{O}(t)}\right)\\
\Rightarrow & max\left\{ \frac{\stackrel[i=0]{(h-1)t-1}{\mathop{\sum}}p^{(i)}(q)}{q^{\left(\alpha h\right)t^{2}}}\right\}  & \in\left(\frac{q^{\left[\left(h-1\right)\left(\alpha+1\right)\right]t^{2}+\mathcal{O}(t)}}{q^{\left(\alpha h\right)t^{2}}}\right)
\end{eqnarray*}
\[
\Rightarrow p\leq\Theta\left(q^{\left(h-\alpha\right)t^{2}+\mathcal{O}(t)}\right)
\]
\end{proof}
\begin{lem}
Let $Pr\left[\mathcal{E}_{i}\right]=Pr\left[rk\left[\begin{array}{c}
\boldsymbol{A}_{j}^{\left(r_{1}\right)}\\
\vdots\\
\boldsymbol{A}_{j}^{\left(r_{h-\epsilon}\right)}
\end{array}\right]<(h-1)t\right]\leq p,\forall1\leq r_{1}<\ldots<r_{h-\epsilon}\leq r$ and $\boldsymbol{A}_{j}^{\left(r_{1}\right)},\ldots,\boldsymbol{A}_{j}^{\left(r_{h-\epsilon}\right)}\in\ensuremath{\mathbb{F}}_{q}^{t\times ht}$,
and each event $\mathcal{E}_{i}$ is independent of all others except
for at most $d$ of them, then $d\leq\frac{\alpha}{\left(\alpha-1\right)!}r^{^{\alpha-1}}$\label{lem:d_e1l1}.
\end{lem}
\begin{proof}
Similar with $\left(\epsilon=1,\ell=1\right)-\ensuremath{N}_{3,r,4}$,
we have:
\[
d\leq\alpha\left(\begin{array}{c}
r-1\\
\alpha-1
\end{array}\right)=\alpha\frac{\left(r-1\right)\ldots\left(r-\alpha+1\right)}{\left(\alpha-1\right)!}\leq\frac{\alpha}{\left(\alpha-1\right)!}r^{^{\alpha-1}}
\]
\end{proof}
\begin{thm}
If $r\leq\Omega\left(q^{\frac{h-\alpha-1}{1-\alpha}t^{2}+\mathcal{O}(t)}\right)$,
then there exists a vector solution for the $\left(\epsilon=1,\ell=1\right)-\ensuremath{N}_{h,r,s}$
network.
\end{thm}
\begin{proof}
As previous, satisfying LLL \ref{thm:LLL} is equivalent to an existence
of a vector solution. We need $4dp\leq1$, and following to Lemma
\ref{lem:d_e1l1}: $d\leq\frac{\alpha}{\left(\alpha-1\right)!}r^{^{\alpha-1}}\Rightarrow r\leq\left(\frac{\left(\alpha-1\right)!}{4\alpha}\cdot\frac{1}{p}\right)^{\frac{1}{\alpha-1}}=r_{max,vector}$.
We thus again find a lower bound of $r_{max,vector}$.

By Lemma \ref{lem:d_e1l1}, we have $p\leq\Theta\left(q^{\left(h-\alpha\right)t^{2}+\mathcal{O}(t)}\right),\forall t\geq2$,

$\Rightarrow min\left\{ r_{max,vector}\right\} \in\Omega\left(q^{\frac{h-\alpha-1}{1-\alpha}t^{2}+\mathcal{O}(t)}\right)$.

Hence, the Local lemma \ref{thm:LLL} is satisfied, when $r\leq\Omega\left(q^{\frac{h-\alpha-1}{1-\alpha}t^{2}+\mathcal{O}(t)}\right)$,
and a vector solution exists for such $r$.
\end{proof}

\subsection{Find the Upper Bound of $r_{max,scalar}$}

\noindent\fbox{\begin{minipage}[t]{1\columnwidth - 2\fboxsep - 2\fboxrule}%
Find $\left(\alpha+1\right)$ received vectors that span a subspace
of dimension $h$. This implies that the $\alpha$ links from the
middle layer carry $\alpha$ vectors which span a subspace of $\ensuremath{\mathbb{F}}_{q_{\mathrm{s}}}^{h}$
whose dimension is at least $\left(h-1\right)$, with $q_{\mathrm{s}}=q^{t}$.%
\end{minipage}}

Following to Theorem (\ref{nw_parameters}), we are interested in
the following range: $\ell+\epsilon+1\leq h\leq\alpha\ell+\epsilon$.

For $3\leq\alpha<h$: all $\alpha$ links must be distinct $\Rightarrow r\leq\left[\begin{array}{c}
\alpha\\
1
\end{array}\right]_{q_{\mathrm{s}}}\Rightarrow r\leq\mathcal{O}\left(q_{s}^{\alpha-1}\right)$

For $\alpha\geq h\geq3$: to achieve $(h-1)$-subspaces of $\ensuremath{\mathbb{F}}_{q_{\mathrm{s}}}^{h}$,
no $\alpha$ links will contain a vector which is contained in the
same $(h-2)$-subspace.

Hence,
\[
r_{max,scalar}\leq\left(\alpha-1\right)\left[\begin{array}{c}
\alpha\\
h-2
\end{array}\right]_{q_{\mathrm{s}}}\Rightarrow r_{max,scalar}\in\mathcal{O}\left(q_{\mathrm{s}}^{\left(\alpha-h+2\right)\left(h-2\right)t^{2}}\right)
\]


\subsection{Calculate Gap }

\begin{eqnarray*}
 & r_{max,scalar} & =min\left\{ r_{max,vector}\right\} \\
\Leftrightarrow & q_{\mathrm{s}}^{\left(\alpha-h+2\right)\left(h-2\right)t^{2}} & =q^{\frac{h-\alpha-1}{1-\alpha}t^{2}+\mathcal{O}(t)}\\
\Leftrightarrow & q_{\mathrm{s}} & ^{=}q^{\frac{\alpha-h+1}{\left(\alpha-1\right)\left(\alpha-h+2\right)\left(h-2\right)}t^{2}+\mathcal{O}(t)}\\
\Rightarrow & g & =q_{\mathrm{s}}-q_{v}=q^{\frac{\alpha-h+1}{\left(\alpha-1\right)\left(\alpha-h+2\right)\left(h-2\right)}t^{2}+\mathcal{O}(t)}
\end{eqnarray*}


\section{$\left(\epsilon=1,\ell\protect\geq2\right)-\mathcal{N}_{h=2\ell,r,s=2\ell+1}$}
\begin{lem}
\label{lem:p_e1l2} Let $Pr\left[\mathcal{E}_{i}\right]=Pr\left[rk\left[\begin{array}{c}
\boldsymbol{A}_{j}^{\left(r_{1}\right)}\\
\vdots\\
\boldsymbol{A}_{j}^{\left(r_{h-\epsilon}\right)}
\end{array}\right]<\left(2\ell-1\right)t\right]\leq p,\forall1\leq r_{1}<\ldots<r_{h-\epsilon}\leq r$ and $\boldsymbol{A}_{j}^{\left(r_{1}\right)},\ldots,\boldsymbol{A}_{j}^{\left(r_{h-\epsilon}\right)}\in\ensuremath{\mathbb{F}}_{q}^{t\times2\ell t}$,
then,
\[
p\leq\Theta\left(q^{-t^{2}-2t-1}\right),\forall t\geq2.
\]
\end{lem}
\begin{proof}
A bad event in this vector network coding problem has the following
probability:
\begin{eqnarray}
Pr\left[rk\left[\begin{array}{c}
\boldsymbol{A}_{j}^{\left(r_{1}\right)}\\
\vdots\\
\boldsymbol{A}_{j}^{\left(r_{h-\epsilon}\right)}
\end{array}\right]<\left(2\ell-1\right)t\right] & = & \stackrel[i=0]{\left(2\ell-1\right)t-1}{\mathop{\sum}}Pr\left[rk\left[\begin{array}{c}
\boldsymbol{A}_{j}^{\left(r_{1}\right)}\\
\vdots\\
\boldsymbol{A}_{j}^{\left(r_{h-\epsilon}\right)}
\end{array}\right]=i\right]\nonumber \\
 & \overset{1}{=} & \stackrel[i=0]{\left(2\ell-1\right)t-1}{\mathop{\sum}}\frac{N_{i,m,n}}{q^{m\cdot n}}\nonumber \\
 & \overset{2}{=} & \frac{1}{q^{4\ell^{2}t^{2}}}\stackrel[i=0]{\left(2\ell-1\right)t-1}{\mathop{\sum}}\stackrel[j=0]{i-1}{\mathop{\prod}}\frac{\left(q^{2\ell t}-q^{j}\right)^{2}}{q^{i}-q^{j}}\label{eq:p_product_e1l2}
\end{eqnarray}

(1): The formula for the umber of $\left[m\times n\right]$ matrices
of rank $i$ over $\ensuremath{\mathbb{F}}_{q}$ was proved in \cite{Overbeck:2007}.

(2): $s=\alpha\ell+\epsilon$ by definition in Section \ref{sec:Description_GCN}
$\Rightarrow\alpha=2$, so $\boldsymbol{A}_{j}\in\ensuremath{\mathbb{F}}_{q}^{2\ell t\times2\ell t}$
with $\boldsymbol{A}_{j}=\left[\begin{array}{c}
\boldsymbol{A}_{j}^{\left(r_{1}\right)}\\
\vdots\\
\boldsymbol{A}_{j}^{\left(r_{h-\epsilon}\right)}
\end{array}\right]$, and $\boldsymbol{A}_{j}$ contains$\left(2\ell\right)$ $t$-dimensional
subspaces of $\ensuremath{\mathbb{F}}_{q}^{2\ell t}$.

We consider the product in Equation \ref{eq:p_product_e1l2}: $\stackrel[j=0]{i-1}{\mathop{\prod}}\frac{\left(q^{2\ell t}-q^{j}\right)^{2}}{q^{i}-q^{j}}=\frac{p_{N}^{(i)}(q)}{p_{D}^{(i)}(q)}=p^{(i)}(q)$.

Due to $i$-times product and large $t$: $\left.\begin{array}{c}
deg\left(p_{N}^{(i)}(q)\right)=q^{i4\ell t}\\
deg\left(p_{D}^{(i)}(q)\right)=q^{i^{2}}
\end{array}\right\} \Rightarrow p^{(i)}(q)\approx q^{i4\ell t-i^{2}}$.

Therefore, we have: $\stackrel[i=0]{\left(2\ell-1\right)t-1}{\mathop{\sum}}\stackrel[j=0]{i-1}{\mathop{\prod}}\frac{\left(q^{2\ell t}-q^{j}\right)^{2}}{q^{i}-q^{j}}=\stackrel[i=0]{\left(2\ell-1\right)t-1}{\mathop{\sum}}p^{(i)}(q)\approx\stackrel[i=0]{\left(2\ell-1\right)t-1}{\mathop{\sum}}q^{i4\ell t-i^{2}}$.

To maximize the sum, we set derivation of it to 0 and find the corresponding
root: 
\begin{eqnarray*}
 & \left(i4\ell t-i^{2}\right)^{'} & =0\\
\Leftrightarrow & 4\ell t-2i & =0\\
\Leftrightarrow & i & =2\ell t
\end{eqnarray*}
However, the upper limit of the sum is $\left(2\ell-1\right)t-1$,
which is less than $2\ell t$ for all $t\geq2$.

\[
\Rightarrow max\left\{ q^{i4\ell t-i^{2}}:i=0,2\ldots,\left(2\ell-1\right)t-1\right\} =\left.q^{i4\ell t-i^{2}}\right|_{i=\left(2\ell-1\right)t-1}=q^{4\ell^{2}t^{2}-t^{2}-2t-1}
\]
Hence, by using the exact bound $\Theta$, we have:
\[
\underset{i}{max}\left\{ \stackrel[i=0]{\left(2\ell-1\right)t-1}{\mathop{\sum}}p^{(i)}(q)\right\} \in\Theta\left(q^{4\ell^{2}t^{2}-t^{2}-2t-1}\right)
\]
\[
\Rightarrow\underset{i}{max}\left\{ \frac{1}{q^{4\ell^{2}t^{2}}}\stackrel[i=0]{\left(2\ell-1\right)t-1}{\mathop{\sum}}p^{(i)}(q)\right\} \in\Theta\left(q^{-t^{2}-2t-1}\right)
\]
\[
\Rightarrow p\leq\Theta\left(q^{-t^{2}-2t-1}\right)
\]
\end{proof}
\begin{lem}
Let $Pr\left[\mathcal{E}_{i}\right]=Pr\left[rk\left[\begin{array}{c}
\boldsymbol{A}_{j}^{\left(r_{1}\right)}\\
\vdots\\
\boldsymbol{A}_{j}^{\left(r_{h-\epsilon}\right)}
\end{array}\right]<\left(2\ell-1\right)t\right]\leq p,\forall1\leq r_{1}<\ldots<r_{h-\epsilon}\leq r$ and $\boldsymbol{A}_{j}^{\left(r_{1}\right)},\ldots,\boldsymbol{A}_{j}^{\left(r_{h-\epsilon}\right)}\in\ensuremath{\mathbb{F}}_{q}^{t\times2\ell t}$,
and each event $\mathcal{E}_{i}$ is independent of all others except
for at most $d$ of them, then $d\leq2r$. \label{lem:d_e1l2}
\end{lem}
\begin{proof}
Being similar with the previous subsections, we have: 
\[
d\leq\alpha\left(\begin{array}{c}
r-1\\
\alpha-1
\end{array}\right)=2\frac{\left(r-1\right)\ldots\left(r-1\right)}{1!}\leq2r
\]
\end{proof}
\begin{thm}
If $r\leq\Omega\left(q^{t^{2}+\mathcal{O}\left(t\right)}\right)$,
then there exists a vector solution for the $\left(\epsilon=1,\ell\geq2\right)-\mathcal{N}_{h=2\ell,r,s=2\ell+1}$
network. \label{theo:r_for_e1l2}
\end{thm}
\begin{proof}
As previous, we need $4\cdot p\cdot d(r)\leq1,\forall r\leq r_{max,vector}$
so that a vector solution exists. Following to Lemma \ref{lem:d_e1l2},
we have $d\leq2r\Rightarrow4\cdot p\cdot2r\leq1\Rightarrow r\leq\frac{1}{8p}$.
We still need to maximize $p$ to get lower on $r_{max,vector}$,
and we have $p\leq\Theta\left(q^{-t^{2}-2t-1}\right),\forall t\geq2$
in Lemma \ref{lem:p_e1l2}. Thus, $min\left\{ r_{max,vector}\right\} \in\Omega\left(\frac{1}{8p}\right)=\Omega\left(q^{t^{2}+2t+1}\right)$.

Hence, the Local lemma in \ref{thm:LLL} is satisfied, when $r\leq\Omega\left(q^{t^{2}/2+\mathcal{O}\left(t\right)}\right)$.
None of bad events occurs, so there exists a vector solution for such
$r$.
\end{proof}
\begin{lem}
A scalar solution for the $\left(\epsilon=1,\ell\geq2\right)-\mathcal{N}_{h=2\ell,r,s=2\ell+1}$
network exists, if and only if there exists a Grasmannian code $\mathcal{G}_{q}\left(h=2\ell,\ell\right)$
such that any $\alpha=2$ subspaces of the set span a subspace of
dimension at least $2\ell-1$. 
\end{lem}
\begin{proof}
Let's denote any 2 subspaces of $\mathcal{G}_{q}\left(h=2\ell,\ell\right)$
as $\mathcal{U}$ and $\mathcal{V}$. If $\mathcal{U}$ and $\mathcal{V}$
span a subspace of dimension at least $2\ell-1$, then we have $dim\left(\mathcal{U}+\mathcal{V}\right)=2\ell-1$,
with $\mathcal{U}+\mathcal{V}$. Therefore, $dim\left(\mathcal{U}\cap\mathcal{V}\right)=dim\left(\mathcal{U}\right)+dim\left(\mathcal{V}\right)-dim\left(\mathcal{U}+\mathcal{V}\right)=1$,
which leads to the subspace distance $d_{s}(\mathcal{U},\mathcal{V})=2dim\left(\mathcal{U}+\mathcal{V}\right)-dim\left(\mathcal{U}\right)-dim\left(\mathcal{V}\right)=2\ell-2$.
No 2 $\ell$-dimensional subspaces of $\ensuremath{\mathbb{F}}_{q_{s}}^{2\ell}$
will contain a vector which is contained in the same $\left(2\ell-2\right)$-subspace,
but $\left(2\ell-1\right)$ of such subspaces can have such vectors,
\[
\Rightarrow r_{scalar}\leq\left(2\ell-1\right)\left[\begin{array}{c}
2\ell\\
2\ell-2
\end{array}\right]_{q_{s}}\Rightarrow r_{scalar}\leq\mathcal{O}\left(q_{s}^{l}\right)
\]
\end{proof}
\begin{cor}
The $\left(\epsilon=1,\ell\geq2\right)-\mathcal{N}_{h=2\ell,r,s=2\ell+1}$
network has a vector solution with a gap $q^{t^{2}/4+\mathcal{O}(t)}$.
\end{cor}
Following to Section \ref{subsec:Comparison-between-scalar-and-vector-sol}
and Theorem \ref{theo:r_for_e1l2}, we have the gap size 
\begin{eqnarray}
 & r_{max,scalar} & =min\left\{ r_{max,vector}\right\} \nonumber \\
\Leftrightarrow & q_{\mathrm{s}}^{\ell} & =q^{t^{2}/2+\mathcal{O}(t)}\nonumber \\
\Leftrightarrow & q_{\mathrm{s}} & ^{=}q^{t^{2}/2\ell+\mathcal{O}(t)}\nonumber \\
\Rightarrow & g & =q_{\mathrm{s}}-q_{v}=q^{t^{2}/2\ell+\mathcal{O}(t)}\label{eq:gap_e1l2}
\end{eqnarray}

This shows us that there exists a better vector solution by comparison
with the gap in \cite[Fig. 4]{Wachter-Zeh:2018}.

\clearpage
    %%%%%%%%%%%%%%%%%%%%%%%%%%%%%%%%%%%%%%%%%%%%%%%
\chapter{Computational Results} \label{chap:comp_method}
%%%%%%%%%%%%%%%%%%%%%%%%%%%%%%%%%%%%%%%%%%%%%%%

In Table \ref{tab:r_over_t}, our vector solutions are computed by
Algorithm \ref{alg:Increasing-Method} for the $\left(\epsilon=1,\ell=1\right)-\ensuremath{N}_{3,r,4}$
network regarding to $t=2$ and $t=3$, with $q_{v}=q^{t}$ as previous
sections. Both construction 1 and 2 provide better results than scalar
solutions. 

Construction 1: $\begin{array}{c|c}
\boldsymbol{I}_{t} & \boldsymbol{T}\end{array}$, with $\boldsymbol{T}\in\ensuremath{\mathbb{F}}_{q}^{t\times t\left(h-1\right)}$

Construction 2: $\boldsymbol{T}\in MatrixSpaceUrs\left(t,3t\right)$

\section{Main Approach \label{sec:Main-Approach}}

Regarding to $t=2$, for a scalar network coding solution we need
a $3-\left(3,1,1\right)_{4}^{c}$ code ($q_{v}=2^{2}=4)$ by Theorem
\ref{theo:scalar_sol_exist}. The largest such code consists of the
21 one-dimensionall subspaces of $\ensuremath{\mathbb{F}}_{4}^{3}$,
each one is contained twice in the code. Therefore, the number of
nodes can be at most 42 for a scalar linear coding solution, while
for vector network coding 89 nodes can be used, i..e. $\mathcal{A}_{q=2}\left(n=6,k=4,\mathrm{t}=3;\lambda=2\right)\geq89$
following to Corollary \ref{cor:dual_subspaces}. This is a new lower
bound for $\mathcal{A}_{2}\left(6,4,3;2\right)$ compared to a code
with 51 codewords presented in \cite{Wachter-Zeh:2018}. The smallest
alphabet size for a scalar solution with 89 nodes exists is $q_{s}=8$.
By Equation \ref{eq:r_scalar_max}, there are 73 one-dimensional subspaces
of $\ensuremath{\mathbb{F}}_{8}^{3}$, and each one can be used twice
in the code; therefore, we have in total 146 possible codesword, but
only 89 codewords are required. In this case, the gap size $g=q_{s}-q_{v}=2^{3}-2^{2}=8-4=2^{2},$i.e.
we achieve a gap size $q^{t^{2}/2}$, which is better the asymptotic
behavior in \ref{eq:gap_e1l1h3rs4}. 

Before introducing the algorithm used to generate such 89 codewords,
we have to state some definitions for a beter explanation. We denote
a matrix space of dimension $\left[n\times m\right]$ by $M(n,m)=\left\{ \boldsymbol{A}:\boldsymbol{A}\in\ensuremath{\mathbb{F}}_{q=2}^{n\times m}\right\} $.
\begin{defn}[Matrix Space with Unique Row Space]
 $U(n,m)$ is a subset of $M(n,m)$, where any $\boldsymbol{A},\boldsymbol{B}\in U(n,m)$
have their row spaces such that:

\[
\mathcal{R}_{q}\left(\boldsymbol{A}\right)\neq\mathcal{R}_{q}\left(\boldsymbol{B}\right),
\]

where $\mathcal{R}_{q}\left(.\right)$ denotes the row space of a
matrix.
\end{defn}
%
\begin{defn}[Sufficient Global Coding Vector]
 Let $\boldsymbol{A},\boldsymbol{B},\boldsymbol{C}\in U(t,3t)$ and
$N\lll\left(\begin{array}{c}
2^{3t^{2}}\\
3
\end{array}\right)$, then a set $\mathcal{H}_{n=3t,m=3t}=\left\{ \boldsymbol{H}_{1},\ldots,\boldsymbol{H}_{N}\right\} $
contains distinct \textit{sufficient global coding vectors} $\boldsymbol{H}_{j}$
such that:

\[
rk\left[\boldsymbol{H}_{j}\right]=rk\left[\begin{array}{c}
\boldsymbol{A}\\
\boldsymbol{B}\\
\boldsymbol{C}
\end{array}\right]\geq2n,\forall j\in\left\{ 1,\ldots,N\right\} .
\]
\end{defn}
For ease of notations, we denote $\left[\begin{array}{c}
\boldsymbol{A}\\
\boldsymbol{B}\\
\boldsymbol{C}
\end{array}\right]$ by $\left\{ \boldsymbol{A},\boldsymbol{B},\boldsymbol{C}\right\} $
for the rest of this thesis.

\begin{defn}[Relative]
 Let $\boldsymbol{A},\boldsymbol{B},\boldsymbol{C}\in U(t,3t)$.
Then $\boldsymbol{C}$ is called a \textit{relative} of a tuple $\left(\boldsymbol{A},\boldsymbol{B}\right)$
if $\left\{ \boldsymbol{A},\boldsymbol{B},\boldsymbol{C}\right\} \in\mathcal{H}_{n=3t,m=3t}$
and denoted as following:

\[
rel\left[\left(\boldsymbol{A},\boldsymbol{B}\right)\right]=\boldsymbol{C}.
\]
\end{defn}
\begin{lem}
There are maximum $3\cdot\left(\begin{array}{c}
2^{3t^{2}}\\
3
\end{array}\right)$ relatives for a set $\mathcal{H}_{n=3t,m=3t}=\left\{ \boldsymbol{H}_{1},\ldots,\boldsymbol{H}_{N}\right\} $.
\label{lem:num_of_relatives}
\end{lem}
\begin{proof}
Each sufficient global coding vector $\boldsymbol{H}_{j}$ contains
a $\left\{ \boldsymbol{A},\boldsymbol{B},\boldsymbol{C}\right\} $.
Thus, each of these 3 matrices can form 3 different relatives:

\[
\begin{array}{c}
rel\left[\left(\boldsymbol{A},\boldsymbol{B}\right)\right]=\boldsymbol{C}\\
rel\left[\left(\boldsymbol{A},\boldsymbol{C}\right)\right]=\boldsymbol{B}\\
rel\left[\left(\boldsymbol{B},\boldsymbol{D}\right)\right]=\boldsymbol{A}
\end{array}.
\]

Furthermore, the maximum size of $\mathcal{H}_{n=3t,m=3t}$ is $q^{3t^{2}}$.
Therefore, $\mathcal{H}_{t=3}$ can have only maximum $3q^{3t^{2}}$
relatives.
\end{proof}
\begin{defn}[Sub-relative]
 Let $\boldsymbol{A},\boldsymbol{B},\boldsymbol{C},\boldsymbol{D}\in U(t,3t)$.
Then $\boldsymbol{D}$ is called a \textit{sub-relative} of a tuple
$\left(\boldsymbol{A},\boldsymbol{B},\boldsymbol{C}\right)\in\mathcal{H}_{n=3t,m=3t}$
if:

\[
\left\{ \begin{array}{c}
\left\{ \boldsymbol{A},\boldsymbol{B},\boldsymbol{D}\right\} \in\mathcal{H}_{t=3}\\
\left\{ \boldsymbol{A},\boldsymbol{C},\boldsymbol{D}\right\} \in\mathcal{H}_{t=3}\\
\left\{ \boldsymbol{B},\boldsymbol{C},\boldsymbol{D}\right\} \in\mathcal{H}_{t=3}
\end{array}\right..
\]

A sub-relative is denoted by: 
\[
subrel\left[\left(\boldsymbol{A},\boldsymbol{B},\boldsymbol{C}\right)\right]=\boldsymbol{D}
\]

This definition about Sub-relative is similarly again used for a set
of 5 or more matrices.
\end{defn}
\begin{algorithm}[H]
\caption{Increasing Method \label{alg:Increasing-Method}}

\hspace*{\algorithmicindent} \textbf{Input}: $\mathcal{H}_{n=3t,m=3t}$ with its size $\left|\mathcal{H}_{n=3t,m=3t}\right|=N\lll\left(\begin{array}{c} 2^{3t^{2}}\\ 3 \end{array}\right)$. \\
\hspace*{\algorithmicindent} \textbf{Output}: A $3-\left(3t,t,t\right)_{q=2}^{c}$ code or its orthogonal complement, a $3-\left(3t,2t,t\right)_{q=2}^{m}$ with its code size is maximum under $\mathcal{H}_{n=3t,m=3t}$.  
\begin{algorithmic}[1]
	\FORALL{$\left(\textbf{A},\textbf{B},\textbf{C}\right) \in \mathcal{H}_{n=3t,m=3t}$}
    	\STATE {$R\leftarrow$ Generate all relatives; ({*})}
	\ENDFOR

	\STATE {$P_{max}\leftarrow\{\}$ // an empty set}
	\FORALL{$\left(\textbf{X},\textbf{Y}\right) \in R$} 
		\IF{$\left|rel\left[\left(\textbf{X},\textbf{Y}\right)\right]\right|$ is maximum}
			\STATE {$P_{max}\leftarrow P_{max}\oplus\left(\textbf{X},\textbf{Y}\right)$ ({**})} 
		\ENDIF
	\ENDFOR

	\STATE {$P_{potential}\leftarrow\{\}$ // an empty set}
	\FORALL{$\left(\textbf{E},\textbf{Q}\right) \in P_{max}$}
		\IF{$\left|\left(\textbf{E},\textbf{Q}\right)\cup rel\left[\left(\textbf{E},\textbf{Q}\right)\right]\right|\notin P_{potential}$}
			\STATE {$P_{potential}\leftarrow P_{potential}\oplus\left(\textbf{E},\textbf{Q}\right)$}
		\ENDIF
	\ENDFOR
	
	\STATE {$P\leftarrow\{\}$ // an empty set}
	\FORALL{$\left(\textbf{K},\textbf{L}\right) \in P_{potential}$}
		\FORALL{$\textbf{Z} \in rel\left[\left(\textbf{X},\textbf{Y}\right)\right]$}
			\IF{$subrel\left[\left(\boldsymbol{K},\boldsymbol{L},\boldsymbol{Z}\right)\right]$ is maximum}
				\STATE {$P\leftarrow P_{potential}\oplus\left(\textbf{K},\textbf{L},\textbf{Z}\right)$ ({***})}
			\ENDIF
		\ENDFOR 
	\ENDFOR
	
	\WHILE{$\exists subrel$ in $P$ is not empty ({****})}
		\STATE {Line 10 to Line 23, but elements of $P_{max}$ is replaced by $P$ and the size of subrel or tuples are thus increased over each while loop.}
	\ENDWHILE
\end{algorithmic}
\end{algorithm}

({*}): If any 2-tuple exists already in $R$, its new relative is
appended to its existing relatives to form a set of relatives.

({*}{*}): $P_{max}$ is set to an empty set anytime we find a new
maximum, before the concatenation $\oplus$.

({*}{*}{*}): $P$ is set to an empty set anytime we find a new maximum,
before the concatenation $\oplus$. $P$ removes duplicated values
itself under the while-loop of Line 11 to Line 15.

({*}{*}{*}{*}): This means the maximum size of subrel in $P$ is not
0.

~

Now, we prove that Algorithm \ref{alg:Increasing-Method} always give
us our expected output. Then we compute its complexity based on number
of necessary computation.
\begin{proof}
Our problem can be described like finding all matrices such that any
3 of them can accompany together. The companion is possible, when
the 3 matrices satisfy the rank requirement in \ref{eq:rk_rqm_e1l1h3s4}.
The input $\mathcal{H}_{n=3t,m=3t}$ of Algorithm \ref{alg:Increasing-Method}
lists all such 3 matrices by tuples of 3 matrices (called 3-tuples).
Line 1 to Line 3 gives us all pairs or 2-tuples from these 3-tuples
as well as their relatives. Then the algorithm starts searching for
its output initially based on 2 matrices having the largest number
of relatives, i.e. it appears mostly in $\mathcal{H}_{n=3t,m=3t}$
in comparison with any other pairs of matrices. This number of relatives
plus 2 by themselves is actually the upper bound for a code size,
when a $3-\left(3t,t,t\right)_{q=2}^{c}$ code is considered \cite[Sec. V-D]{Zhang:2019}.
Furthermore, from Line 10 to Line 26, the algorithm extends the inital
2-tuples based on all of these relatives. That's why the algorithm's
output gives us all of our expected matrices.
\end{proof}
Our algorithm is based on $C_{relatives}\lll3\left(\begin{array}{c}
2^{3t^{2}}\\
3
\end{array}\right)\Rightarrow C_{relatives}\lll\mathcal{O}\left(\frac{t^{6}}{6}\right)$. In case $t=2$, $U(t,3t)$ helps us reduce $\left|M(t,3t)\right|=4096$
matrices to $\left|U(t,3t)\right|=715$ matrices, $C_{relatives}\approx3\left(\begin{array}{c}
\frac{715}{4096}2^{3t^{2}}\\
3
\end{array}\right)\approx\mathcal{O}\left(\frac{2}{3000}t^{6}\right)$. Therefore, we are able to achieve the output of Algorithm \ref{alg:Increasing-Method}
for $\mathcal{H}_{n=6,m=6}$ in a short amount of time.

We are going to list some toy examples for the algorithm as well as
illustrate it in Figure \ref{fig:rel_example}.
\begin{example}
Let $n=1,m=2,q=2$, which violates the input $\mathcal{H}_{n=3t,m=3t}$
but still being a good example for the rest steps in Algorithm \ref{alg:Increasing-Method}.
Then we have $N=4$ matrices (vectors):
\[
\begin{array}{c}
\boldsymbol{A}=[0,0]\\
\boldsymbol{B}=[0,1]\\
\boldsymbol{C}=[1,0]\\
\boldsymbol{D}=[1,1]
\end{array}
\]

\uline{Step 1} (Line 1 to Line 3): Due to, any 3 of them form a
sufficient global coding vector, i.e. a matrix with $rk\geq2n$, we
have the relative as following:
\[
\begin{array}{c}
rel\left[\left(\boldsymbol{A},\boldsymbol{B}\right)\right]=[\boldsymbol{C},\boldsymbol{D}]\\
rel\left[\left(\boldsymbol{A},\boldsymbol{C}\right)\right]=[\boldsymbol{B},\boldsymbol{D}]\\
rel\left[\left(\boldsymbol{A},\boldsymbol{D}\right)\right]=[\boldsymbol{B},\boldsymbol{C}]\\
rel\left[\left(\boldsymbol{B},\boldsymbol{C}\right)\right]=[\boldsymbol{A},\boldsymbol{D}]\\
rel\left[\left(\boldsymbol{B},\boldsymbol{D}\right)\right]=[\boldsymbol{A},\boldsymbol{C}]\\
rel\left[\left(\boldsymbol{C},\boldsymbol{D}\right)\right]=[\boldsymbol{A},\boldsymbol{B}]
\end{array}
\]

\uline{Step 2} (Line 4 to Line 9): We get the maximum size of these
steps is $2$ and $P_{max}$ results in all $\left\{ \boldsymbol{A},\boldsymbol{B}\right\} ,\left\{ \boldsymbol{A},\boldsymbol{C}\right\} ,\left\{ \boldsymbol{A},\boldsymbol{D}\right\} ,\left\{ \boldsymbol{B},\boldsymbol{C}\right\} ,\left\{ \boldsymbol{B},\boldsymbol{D}\right\} ,\left\{ \boldsymbol{C},\boldsymbol{D}\right\} $,
because $\underset{\begin{array}{c}
\forall\boldsymbol{X},\boldsymbol{Y}\in M(1,2)\\
\boldsymbol{X}\neq\boldsymbol{Y}
\end{array}}{max}\left(rel\left[\left(\boldsymbol{X},\boldsymbol{Y}\right)\right]\right)=2$.

\uline{Step 3} (Line 10 to Line 14): Due to $\left\{ \boldsymbol{E},\boldsymbol{Q}\right\} \cup rel\left[\left(\boldsymbol{E},\boldsymbol{Q}\right)\right]=\left\{ \boldsymbol{A},\boldsymbol{B},\boldsymbol{C},\boldsymbol{D}\right\} $
with $\left(\boldsymbol{E},\boldsymbol{Q}\right)$ are tuples from
$P_{max}$ found in Step 2. We keep only $P_{potential}=\left\{ \left(\boldsymbol{A},\boldsymbol{B}\right):rel\left[\left(\boldsymbol{A},\boldsymbol{B}\right)\right]\right\} $

\uline{Step 4} (Line 16 to 23): Regarding to $rel\left[\left(\boldsymbol{K},\boldsymbol{L}\right)\right]=rel\left[\left(\boldsymbol{A},\boldsymbol{B}\right)\right]$,
we have $\boldsymbol{Z}\in\left\{ \boldsymbol{C},\boldsymbol{D}\right\} $.
Then, $\left|subrel\left[\left(\boldsymbol{A},\boldsymbol{B},\boldsymbol{C}\right)\right]\right|=\left|subrel\left[\left(\boldsymbol{A},\boldsymbol{B},\boldsymbol{D}\right)\right]\right|=1$,
so we get $P=\left\{ \left(\boldsymbol{A},\boldsymbol{B},\boldsymbol{C}\right),\left(\left(\boldsymbol{A},\boldsymbol{B},\boldsymbol{D}\right)\right)\right\} $.

\uline{Step 5}: The maximum size of subrel in $P$ is larger than
$0$, so we proceed the while loop. Step 3 will be looped firstly,
and we get $P_{potential}=\left\{ \left(\boldsymbol{A},\boldsymbol{B},\boldsymbol{C}\right):rel\left[\left(\boldsymbol{A},\boldsymbol{B},\boldsymbol{C}\right)\right]\right\} $.

\uline{Step 6}: As step 4, we get a result $\left\{ \boldsymbol{A},\boldsymbol{B},\boldsymbol{C},\boldsymbol{D}\right\} $.
\end{example}
%
\begin{example}
For further understading, we use Figure \ref{fig:rel_example} for
illustration. In the right, we observe that the size of relative becomes
smaller when its tuple identity is larger, i.e. $\left|rel\left[\left(\boldsymbol{A},\boldsymbol{B}\right)\right]\right|\geq\left|subrel\left[\left(\boldsymbol{A},\boldsymbol{B},\boldsymbol{C}\right)\right]\right|$
or $\left|rel\left[\left(\boldsymbol{A},\boldsymbol{B}\right)\right]\right|\geq\left|subrel\left[\left(\boldsymbol{A},\boldsymbol{B},\boldsymbol{D}\right)\right]\right|$.
It explains why the maximum numbers of matrices that we can find for
a vector solution of this network is the output of Algorithm \ref{alg:Increasing-Method}.
Regarding to the left, the visual explanation of $subrel$ is shown.
\end{example}
\begin{figure}[H]
\caption{The vector network coding of $(\epsilon=1,l=1)-\mathcal{N}_{h=3,r,s=4}$
represents as a matrix problem\label{fig:rel_example}}

\centering{}\includegraphics[width=0.5\paperwidth]{E:/Documents/TUM/THESIS/thesisCOD_Ha/figures/rel_example}
\end{figure}


\section{Alternative Approaches}

Firstly, instead of using \textit{matrix space with unique row space},
we use the whole matrix space $M\left(n,m\right)$. In case $t=2$,
we cover all $2^{3t^{2}}=4096$ matrices instead of only 715 matrices
in Section \ref{sec:Main-Approach}, which will help the Algorithm
\ref{alg:Increasing-Method} cover the optimal vector solution.
\begin{defn}[Good Global Coding Vector\label{def:almost_sufficient_global_coding_vector}]
 Let $\boldsymbol{A},\boldsymbol{B},\boldsymbol{C}\in U(t,3t)$ and
$N\leq\left(\begin{array}{c}
2^{3t^{2}}\\
3
\end{array}\right)$, then a set $\mathcal{T}_{n=3t,m=3t}=\left\{ \boldsymbol{T}_{1},\ldots,\boldsymbol{T}_{N}\right\} $
contains distinct \textit{sufficient global coding vectors} $\boldsymbol{T}_{j}$
such that:

\[
rk\left[\boldsymbol{T}_{j}\right]=rk\left[\begin{array}{c}
\boldsymbol{A}\\
\boldsymbol{B}\\
\boldsymbol{C}
\end{array}\right]\geq2n,\forall j\in\left\{ 1,\ldots,N\right\} .
\]
\end{defn}
In Algorithm \ref{alg:Increasing-Method}, we substitute the input
by $\mathcal{T}_{n=3t,m=3t}$ instead of $\mathcal{H}_{n=3t,m=3t}$.
This will give us an output with upper bound of $\mathcal{A}_{2}\left(3t,2t,3;2t\right)=126$
\cite[Sec. V-B]{Zhang:2019}.

Secondly, we tried another approach called ``Randomly Increasing
Method''. We got about 72 codewords instead of 89 codewords as Algorithm
\ref{alg:Increasing-Method}, due to a limited number of tries.

Finally, we tried removing bad matrices from a set of all 4096 matrices,
until any 3 of the matrices left in the set satisfy the Equation \ref{eq:rk_rqm_e1l1h3s4}.
A matrix with the less number of existence in $\mathcal{T}_{n=3t,m=3t}$
or $\mathcal{H}_{n=3t,m=3t}$ is evaluated as a bad matrix, and it
is removed for each step. If many bad matrices exist, we only remove
the first one. In case $t=2$, we unfortunately got about 69 codewords
instead of 89 codewords as Algorithm \ref{alg:Increasing-Method}.

\clearpage
    %%%%%%%%%%%%%%%%%%%%%%%%%%%%%%%%%%%%%%%%%%%%%%%
\chapter{Conclusion} \label{chap:conclusion}
%%%%%%%%%%%%%%%%%%%%%%%%%%%%%%%%%%%%%%%%%%%%%%%

In this thesis, we have shown combinatorial proofs for an existence
of new gaps for 3 generalized combination networks. The $\left(\epsilon=1,\ell=1\right)-\mathcal{N}_{h=3,r,s=4}$
network with $t=2$ has been studied in \cite{Wachter-Zeh:2018,Etzion:2016,Zhang:2019,Etzion:2018}
but no general gap was found. We also got computational results for
this network with 89 two-dimensional subspaces of $\ensuremath{\mathbb{F}}_{2}^{6}$,
which is better the constructed set of 51 two-dimensional subspaces
of $\ensuremath{\mathbb{F}}_{2}^{6}$ found in \cite{Wachter-Zeh:2018}.

In Chapter 5, by applying the Local lemma, we stated that there is
an $r_{\mathrm{max,vector}}\in\Omega\left(q^{t^{2}/2+\mathcal{O}\left(t\right)}\right)$
such that for any $r\leq r_{\mathrm{max,vector}}$ there exists a
vector solution for the $\left(\epsilon=1,l=1\right)-\mathcal{N}_{h=3,r,s=4}$
network. The optimal scalar solution for such network exists, when
$r\leq\mathcal{O}\left(q_{\mathrm{s}}^{2}\right)$. Therefore, a lower
bound on the gap $g_{\mathrm{lower\,bound}}=q^{t^{2}/4+\mathcal{O}(t)}$
exists for the $\left(1,1\right)-\mathcal{N}_{3,r,4}$ network. Similarly
we derived the gaps for the $\left(\epsilon=1,\ell=1\right)-\mathcal{N}_{h,r,s}$
network, the $\left(\epsilon>1,\ell=1\right)-\mathcal{N}_{h,r,s}$
network and the $\left(\epsilon=1,\ell>1\right)-\mathcal{N}_{h=2\ell,r,s=2\ell+1}$
network, respectively with the following gap 
\[
\begin{array}{c}
g_{\mathrm{lower\,bound}}=q^{\frac{\alpha-h+1}{\left(\alpha-1\right)\left(\alpha-h+2\right)\left(h-2\right)}t^{2}+\mathcal{O}(t)},\\
g_{\mathrm{lower\,bound}}=q^{\frac{\epsilon\left(\alpha-h+\epsilon\right)}{\left(\alpha-1\right)\left(\alpha-h+\epsilon+1\right)\left(h-\epsilon-1\right)}t^{2}+\mathcal{O}(t)},\\
g_{\mathrm{lower\,bound}}=q^{t^{2}/2\ell+\mathcal{O}(t)}.
\end{array}
\]
A comparison with known results can be found in Table \ref{tab:New-gap-found}
and Table \ref{tab:r_over_t}.

In Chapter 6, we introduced 4 different approaches in computing vector
solutions that outperform the optimal scalar solution for the $\left(\epsilon=1,\ell=1\right)-\mathcal{N}_{h=3,r,s=4}$
network with $t=2$. All approaches gave us such vector solutions,
and the Algorithm \ref{alg:Increasing-Method} called ``Increasing
Method'' generated the best result among 4 approaches. The best result
is about 2 times 42 results of the optimal scalar solution, i.e. we
found $r_{vector}=89$. When we mention the results of the optimal
scalar solution, it means that such a solution exists if and only
if $r_{scalar}\leq42$. However, our computational result of $r_{vector}$
is still less than the upper bound of $\mathcal{A}_{2}\left(6,4,3;2\right)$
in \cite{Etzion:2018}. Hence, we can only conclude $89\leq\mathcal{A}_{2}\left(6,4,3;2\right)\leq126$
for the $\left(\epsilon=1,\ell=1\right)-\mathcal{N}_{h=3,r,s=4}$
network with $t=2$. This motivates an open research to find a computational
method for generating a vector solution of 126 two-dimensional subspaces
of $\ensuremath{\mathbb{F}}_{2}^{6}$. At the time writing this thesis,
the best computational result is $r_{vector}=121$ stated in \cite{Etzion:2018}.
In Appendix \ref{sec:89-Two-Dimensional-Subspaces}, we listed one
of 2 different sets of our 89 two-dimensional subspaces of $\ensuremath{\mathbb{F}}_{2}^{6}$,
which are slightly different in 2 subspaces. We also find an interesting
result that there are $\left|U(2,6)\right|=715$ matrices whose different
row spaces among $\left|M(2,6)\right|=4096$ matrices over $\ensuremath{\mathbb{F}}_{2}^{6}$.
Furthermore, we used the Algorithm \ref{alg:Increasing-Method} and
got $r_{vector}=166$ for $t=3$ by Construction 1, which is better
than the optimal scalar solution existing if and only $r_{scalar}\leq146$.
This computational result was not found in any previous studies in
our scope of knowledge. For the $\left(\epsilon=1,\ell=1\right)-\mathcal{N}_{h=3,r,s=4}$
network with $t=3$, we therefore state a new bound $166\leq\mathcal{A}_{2}\left(9,6,3;2\right)\leq537$.
In Appendix \ref{sec:166-Three-Dimensional-Subspaces}, we wrote down
one of 18 found variants of 166 three-dimensional subspaces of $\ensuremath{\mathbb{F}}_{2}^{9}$.
From $\left|M(3,6)\right|=262144$ matrices, we found $\left|U(3,6)\right|=2110$
matrices whose different row spaces. All of computational results
in this study was listed in Table \ref{tab:r_over_t}. 

Another open research is to study a gap for the general network $\left(\epsilon,\ell\right)-\mathcal{N}_{h,r,s}$
with $\left(\epsilon>1,\ell>1\right)$ or $\left(\epsilon=1,\ell>1\right)$
for any $\alpha\geq1$. The most challenging problem for such research
is to define the optimal scalar solution for such network.

\clearpage
        
        \cleardoubleemptypage


% ########################################
% Appendix / Anhang:
% ########################################

% roman page numbering, starting with page number 1 (capitals)/ roemische Seitennummerierung beginnend mit Seite 1 (gross)
    \appendix
    \setcounter{page}{1}
    \pagenumbering{Roman}  
    %%%%%%%%%%%%%%%%%%%%%%%%%%%%%%%%%%%%%%%%%%%%%%%
\chapter{Appendices} \label{chap:appendix}
%%%%%%%%%%%%%%%%%%%%%%%%%%%%%%%%%%%%%%%%%%%%%%%

\section{89 Two-Dimensional Subspaces of $\ensuremath{\mathbb{F}}_{2}^{6}$
for the $\left(\epsilon=1,\ell=1\right)-\mathcal{N}_{h=3,r,s=4}$
Network \label{sec:89-Two-Dimensional-Subspaces}}

\begin{lstlisting}
[1 0 0 0 0 0] [1 0 0 0 0 0] [0 0 1 0 0 0] [0 0 0 1 0 0] 
[0 1 0 0 0 0] [0 0 1 0 0 0] [0 0 0 0 0 1] [0 0 0 0 1 0] 

[1 0 0 0 1 0] [1 0 0 0 0 0] [1 0 0 0 0 0] [0 1 0 0 1 0] 
[0 0 0 0 0 1] [0 0 1 0 1 0] [0 0 0 1 0 1] [0 0 0 1 0 0] 

[0 1 0 0 0 1] [0 0 1 0 0 1] [0 0 0 1 0 1] [1 1 1 0 0 0] 
[0 0 1 0 0 0] [0 0 0 1 0 0] [0 0 0 0 1 0] [0 0 0 0 1 0] 

[1 0 1 0 0 0] [1 0 1 0 0 0] [1 0 1 0 0 0] [1 0 0 1 0 0] 
[1 0 0 1 0 0] [0 1 0 1 0 0] [0 0 0 1 1 0] [1 0 0 0 0 1] 

[1 0 0 0 1 1] [1 0 0 0 1 0] [1 0 0 0 0 1] [0 1 1 1 0 0] 
[0 0 0 1 0 0] [1 0 0 0 0 1] [0 1 0 1 0 0] [0 0 0 0 1 0] 

[0 1 1 0 0 0] [0 1 1 0 0 0] [0 1 0 0 0 1] [0 1 0 0 0 0] 
[0 1 0 0 1 0] [0 0 0 1 0 1] [0 0 1 1 0 0] [0 0 1 1 1 0] 

[0 0 1 1 1 0] [0 0 1 1 0 0] [1 1 1 0 0 0] [1 1 1 0 0 0] 
[0 0 0 0 0 1] [0 0 0 0 1 1] [1 0 0 0 1 0] [0 1 0 1 0 0] 

[1 1 0 1 0 0] [1 1 0 0 1 0] [1 1 0 0 1 0] [1 1 0 0 0 1] 
[0 1 0 0 1 0] [0 1 0 1 0 0] [0 1 0 0 0 1] [1 0 0 1 0 0] 

[1 1 0 0 0 0] [1 1 0 0 0 0] [1 0 1 1 1 0] [1 0 1 1 0 0] 
[1 0 1 0 0 1] [0 0 0 1 1 1] [0 0 0 0 0 1] [0 1 0 0 0 1] 

[1 0 1 1 0 0] [1 0 1 0 1 1] [1 0 1 0 1 1] [1 0 0 1 1 0] 
[0 0 1 0 0 1] [0 1 0 0 0 0] [0 0 0 1 0 0] [0 0 0 0 1 1] 

[1 0 0 0 0 1] [0 1 1 0 0 1] [0 1 0 1 1 1] [0 0 1 1 0 1] 
[0 1 0 1 1 0] [0 0 1 0 1 0] [0 0 1 0 0 0] [0 0 1 0 1 0] 

[0 0 1 1 0 0] [1 1 1 1 1 0] [1 1 1 1 0 0] [1 1 1 0 1 0] 
[0 0 1 0 1 1] [0 0 0 0 0 1] [0 1 0 0 0 1] [0 0 1 0 0 1] 

[1 1 1 0 0 1] [1 1 0 1 1 0] [1 1 0 1 0 1] [1 1 0 1 0 0] 
[0 0 0 1 1 0] [1 0 1 0 0 0] [0 1 1 0 0 0] [1 0 0 0 1 1] 

[1 1 0 1 0 0] [1 1 0 0 1 1] [1 1 0 0 1 1] [1 1 0 0 1 0] 
[0 0 1 1 0 1] [0 1 1 0 0 0] [0 0 1 1 0 0] [0 1 1 1 0 0] 

[1 1 0 0 0 1] [1 1 0 0 0 1] [1 0 1 1 0 0] [1 0 1 1 0 0] 
[1 0 1 0 1 0] [0 1 1 0 1 0] [0 1 0 1 0 1] [0 0 0 1 1 1] 

[1 0 1 0 1 0] [1 0 1 0 0 1] [1 0 1 0 0 1] [1 0 0 1 1 0] 
[0 1 0 1 1 0] [0 1 1 1 0 0] [0 1 0 1 0 1] [0 1 0 1 0 1] 

[1 0 0 1 0 1] [1 0 0 1 0 1] [1 0 0 0 0 1] [0 1 1 1 0 1] 
[0 1 1 0 1 0] [0 0 1 0 1 1] [0 1 1 1 1 0] [0 0 0 1 1 0] 

[0 1 1 0 1 1] [0 1 1 0 0 1] [0 1 0 1 0 1] [0 1 0 0 1 1] 
[0 1 0 1 0 0] [0 1 0 1 1 0] [0 0 1 0 1 1] [0 0 1 1 1 0] 

[1 1 1 1 1 0] [1 1 1 1 1 0] [1 1 1 1 0 0] [1 1 1 0 0 1] 
[0 0 0 1 0 1] [0 0 0 0 1 1] [0 1 0 0 1 1] [0 0 1 1 1 0] 

[1 1 0 1 1 0] [1 1 0 1 1 0] [1 1 0 1 0 1] [1 1 0 1 0 0] 
[1 0 1 0 0 1] [0 0 1 1 0 1] [0 0 1 0 1 1] [0 1 1 0 1 1] 

[1 1 0 0 1 1] [1 1 0 0 0 1] [1 0 1 1 0 1] [1 0 1 0 1 0] 
[0 0 1 1 1 0] [1 0 1 1 1 0] [0 1 1 0 1 0] [0 1 0 1 1 1] 

[1 0 0 1 1 1] [1 0 0 1 1 1] [1 1 1 0 1 0] [1 1 1 0 1 0] 
[0 1 1 1 0 0] [0 1 1 0 1 0] [1 0 1 1 0 1] [1 0 0 1 1 1] 

[1 1 0 1 1 0] 
[1 0 1 1 0 1]
\end{lstlisting}


\section{166 Three-Dimensional Subspaces of $\ensuremath{\mathbb{F}}_{2}^{9}$
for the $\left(\epsilon=1,\ell=1\right)-\mathcal{N}_{h=3,r,s=4}$
Network \label{sec:166-Three-Dimensional-Subspaces}}

\begin{lstlisting}
[1 0 0 0 0 1 0 0 0] [1 0 0 0 0 0 1 0 0] 
[0 1 0 0 0 0 0 1 0] [0 1 0 0 0 0 0 0 1] 
[0 0 1 0 0 0 0 0 0] [0 0 1 0 0 0 0 0 0] 

[1 0 0 1 0 1 0 1 0] [1 0 0 1 1 0 1 0 1] 
[0 1 0 1 0 0 1 0 0] [0 1 0 1 0 1 0 0 0] 
[0 0 1 0 0 0 0 1 1] [0 0 1 0 0 0 0 1 0] 

[1 0 0 1 0 0 1 0 0] [1 0 0 0 0 1 1 1 0] 
[0 1 0 0 1 1 0 1 0] [0 1 0 0 0 0 0 1 1] 
[0 0 1 0 0 0 0 0 1] [0 0 1 0 0 0 0 0 0] 

[1 0 0 1 0 1 0 1 0] [1 0 0 0 0 1 1 0 1] 
[0 1 0 0 1 1 1 0 0] [0 1 0 0 0 0 1 1 0] 
[0 0 1 0 1 0 0 0 1] [0 0 1 0 0 0 0 0 0] 

[1 0 0 1 1 0 0 0 0] [1 0 0 1 0 1 0 1 0] 
[0 1 0 0 0 0 0 1 0] [0 1 0 0 1 0 1 1 0] 
[0 0 1 0 0 0 0 0 0] [0 0 1 0 0 0 1 0 1] 

[1 0 0 1 0 0 0 1 0] [1 0 0 1 0 0 0 0 1] 
[0 1 0 0 0 1 0 0 0] [0 1 0 0 0 0 1 0 0] 
[0 0 1 0 0 0 0 0 0] [0 0 1 0 0 0 0 0 0] 

[1 0 0 1 1 0 1 0 0] [1 0 0 1 0 0 0 0 0] 
[0 1 0 1 0 1 0 1 0] [0 1 0 0 1 0 0 0 1] 
[0 0 1 0 0 0 0 1 1] [0 0 1 0 0 0 0 0 0] 

[1 0 0 1 0 0 0 0 0] [1 0 0 1 0 0 0 0 0] 
[0 1 0 0 1 0 0 0 0] [0 1 0 0 0 1 0 0 0] 
[0 0 1 0 0 1 0 0 0] [0 0 1 0 0 0 1 0 0] 

[1 0 0 1 0 0 0 0 0] [1 0 0 0 1 0 0 1 0] 
[0 1 0 0 0 0 1 0 0] [0 1 0 0 0 0 0 0 1] 
[0 0 1 0 0 0 0 1 0] [0 0 1 0 0 0 0 0 0] 

[1 0 0 1 1 0 0 1 0] [1 0 0 0 1 0 0 0 0] 
[0 1 0 1 0 0 1 0 1] [0 1 0 0 0 1 0 0 0] 
[0 0 1 0 1 1 0 0 1] [0 0 1 0 0 0 1 0 0] 

[1 0 0 0 1 0 0 0 0] [1 0 0 1 1 0 1 0 0] 
[0 1 0 0 0 0 1 1 0] [0 1 0 1 0 0 0 1 0] 
[0 0 1 0 0 0 0 0 0] [0 0 1 1 0 0 0 0 1] 

[1 0 0 0 0 1 0 1 0] [1 0 0 1 1 0 0 0 1] 
[0 1 0 0 0 0 1 0 0] [0 1 0 0 1 1 0 0 0] 
[0 0 1 0 0 0 0 0 0] [0 0 1 0 0 0 1 1 1] 

[1 0 0 1 1 0 1 0 0] [1 0 0 1 0 0 0 1 1] 
[0 1 0 0 1 1 0 0 0] [0 1 0 0 1 0 1 0 0] 
[0 0 1 0 1 0 0 0 1] [0 0 1 0 0 1 0 0 0] 

[1 0 0 1 1 0 1 0 0] [1 0 0 0 0 0 1 0 0] 
[0 1 0 0 0 1 1 1 1] [0 1 0 0 0 0 0 1 0] 
[0 0 1 0 0 0 0 0 0] [0 0 1 0 0 0 0 0 1] 

[1 0 0 0 0 1 0 1 0] [1 0 0 1 1 0 1 1 0] 
[0 1 0 0 0 1 0 0 1] [0 1 0 1 0 1 0 0 0] 
[0 0 1 0 0 0 1 0 0] [0 0 1 0 1 0 0 0 1] 

[1 0 0 1 0 0 0 1 0] [1 0 0 1 1 0 0 1 1] 
[0 1 0 1 0 0 0 0 1] [0 1 0 1 0 0 1 0 0] 
[0 0 1 0 1 1 0 0 0] [0 0 1 0 0 1 0 0 0] 

[1 0 0 1 0 1 0 0 1] [1 0 0 1 1 0 1 1 0] 
[0 1 0 1 0 0 0 1 0] [0 1 0 0 1 1 0 0 0] 
[0 0 1 0 1 0 1 0 1] [0 0 1 0 0 0 1 0 1] 

[1 0 0 1 0 1 0 0 1] [1 0 0 1 0 1 0 0 1] 
[0 1 0 0 1 1 1 0 0] [0 1 0 0 1 1 0 1 0] 
[0 0 1 0 0 0 1 1 0] [0 0 1 0 1 0 1 0 0] 

[1 0 0 1 0 1 0 1 0] [1 0 0 1 1 0 1 0 1] 
[0 1 0 0 1 0 0 0 1] [0 1 0 0 0 0 0 0 0] 
[0 0 1 0 0 1 1 0 0] [0 0 1 0 0 0 0 0 0] 

[1 0 0 1 0 1 0 0 1] [1 0 0 1 1 0 0 1 0] 
[0 1 0 0 1 0 1 0 1] [0 1 0 1 0 1 1 0 0] 
[0 0 1 0 0 1 0 1 0] [0 0 1 0 1 1 0 0 1] 

[1 0 0 1 0 1 0 0 0] [1 0 0 1 1 0 0 1 0] 
[0 1 0 0 1 0 0 0 0] [0 1 0 0 1 1 0 0 0] 
[0 0 1 0 0 0 0 1 0] [0 0 1 0 1 0 1 0 0] 

[1 0 0 1 1 0 0 1 0] [1 0 0 1 0 0 1 0 0] 
[0 1 0 0 1 0 1 0 0] [0 1 0 0 1 1 0 0 0] 
[0 0 1 0 1 0 0 0 1] [0 0 1 0 0 0 0 0 0] 

[1 0 0 1 0 0 0 1 0] [1 0 0 1 1 0 1 0 1] 
[0 1 0 0 1 0 0 0 0] [0 1 0 0 1 1 0 0 0] 
[0 0 1 0 0 0 0 0 1] [0 0 1 0 1 0 0 1 0] 

[1 0 0 1 1 0 0 0 1] [1 0 0 1 1 0 0 0 1] 
[0 1 0 1 0 1 1 1 0] [0 1 0 1 0 1 0 0 0] 
[0 0 1 0 0 0 0 0 0] [0 0 1 1 0 0 1 0 0] 

[1 0 0 1 1 0 0 0 1] [1 0 0 1 0 0 0 1 1] 
[0 1 0 1 0 0 0 1 0] [0 1 0 0 1 1 0 1 0] 
[0 0 1 0 1 1 0 0 0] [0 0 1 0 1 0 1 0 0] 

[1 0 0 1 1 0 0 0 1] [1 0 0 1 1 0 0 0 1] 
[0 1 0 0 1 1 0 1 0] [0 1 0 0 1 0 1 1 0] 
[0 0 1 0 0 0 1 0 0] [0 0 1 0 0 1 0 0 0] 

[1 0 0 0 1 1 0 0 0] [1 0 0 1 0 1 0 0 0] 
[0 1 0 0 1 0 0 0 1] [0 1 0 1 0 0 0 1 1] 
[0 0 1 0 0 0 0 0 0] [0 0 1 0 1 0 1 1 0] 

[1 0 0 0 1 0 1 0 0] [1 0 0 1 1 0 1 0 0] 
[0 1 0 0 0 1 0 0 0] [0 1 0 1 0 1 0 0 0] 
[0 0 1 0 0 0 0 0 1] [0 0 1 0 0 0 1 1 1] 

[1 0 0 1 0 0 0 0 0] [1 0 0 1 1 0 0 0 1] 
[0 1 0 0 1 1 0 1 0] [0 1 0 1 0 1 0 1 0] 
[0 0 1 0 0 1 1 0 0] [0 0 1 0 1 1 1 0 0] 

[1 0 0 1 1 0 0 0 1] [1 0 0 1 0 0 0 0 0] 
[0 1 0 1 0 1 0 1 0] [0 1 0 0 1 0 1 0 0] 
[0 0 1 0 1 0 1 1 0] [0 0 1 0 1 0 0 1 1] 

[1 0 0 1 1 0 0 0 0] [1 0 0 1 1 0 0 0 0] 
[0 1 0 1 0 1 1 0 0] [0 1 0 1 0 1 0 1 0] 
[0 0 1 0 0 1 0 0 1] [0 0 1 0 0 1 1 0 0] 

[1 0 0 1 1 0 0 0 0] [1 0 0 1 0 0 0 0 0] 
[0 1 0 1 0 1 0 0 1] [0 1 0 0 1 1 1 0 1] 
[0 0 1 1 0 0 0 1 0] [0 0 1 0 1 0 0 1 0] 

[1 0 0 0 1 0 0 0 0] [1 0 0 1 0 0 0 0 0] 
[0 1 0 0 0 0 1 0 0] [0 1 0 0 0 1 1 0 1] 
[0 0 1 0 0 0 0 1 1] [0 0 1 0 0 0 1 1 0] 

[1 0 0 1 1 0 0 0 0] [1 0 0 1 0 0 0 0 0] 
[0 1 0 1 0 0 1 0 1] [0 1 0 0 1 1 0 0 1] 
[0 0 1 1 0 0 0 1 0] [0 0 1 0 0 1 1 1 0] 

[1 0 0 0 0 1 1 0 0] [1 0 0 1 0 1 0 0 0] 
[0 1 0 0 0 1 0 1 0] [0 1 0 1 0 0 1 0 1] 
[0 0 1 0 0 0 0 0 0] [0 0 1 0 1 0 0 1 1] 

[1 0 0 1 0 1 0 0 1] [1 0 0 1 0 1 0 0 1] 
[0 1 0 1 0 0 1 1 0] [0 1 0 1 0 0 1 1 0] 
[0 0 1 0 1 1 1 0 0] [0 0 1 0 1 1 0 1 0] 

[1 0 0 0 1 1 1 1 0] [1 0 0 1 1 0 0 1 0] 
[0 1 0 0 0 1 0 0 1] [0 1 0 1 0 1 0 0 1] 
[0 0 1 0 0 0 0 0 0] [0 0 1 0 0 1 1 0 0] 

[1 0 0 1 1 0 0 1 1] [1 0 0 0 1 0 1 0 0] 
[0 1 0 1 0 1 1 0 1] [0 1 0 0 1 0 0 1 1] 
[0 0 1 0 0 0 0 0 0] [0 0 1 0 0 1 0 0 0] 

[1 0 0 0 1 1 1 0 0] [1 0 0 0 0 0 1 1 0] 
[0 1 0 0 0 0 1 1 0] [0 1 0 0 0 0 1 0 1] 
[0 0 1 0 0 0 0 0 1] [0 0 1 0 0 0 0 0 0] 

[1 0 0 0 1 1 0 1 0] [1 0 0 1 0 0 1 1 0] 
[0 1 0 0 0 1 1 0 0] [0 1 0 1 0 0 0 0 1] 
[0 0 1 0 0 0 0 0 1] [0 0 1 0 1 1 0 1 0] 

[1 0 0 1 0 1 1 0 1] [1 0 0 0 1 1 0 0 1] 
[0 1 0 0 1 0 1 1 0] [0 1 0 0 1 0 0 1 0] 
[0 0 1 0 0 0 0 0 0] [0 0 1 0 0 0 1 0 0] 

[1 0 0 0 1 1 0 0 1] [1 0 0 0 1 1 1 0 0] 
[0 1 0 0 0 1 0 1 0] [0 1 0 0 1 0 0 1 0] 
[0 0 1 0 0 0 1 0 0] [0 0 1 0 0 0 1 0 1] 

[1 0 0 0 1 1 0 0 0] [1 0 0 1 0 1 1 0 0] 
[0 1 0 0 1 0 1 0 1] [0 1 0 1 0 0 0 0 1] 
[0 0 1 0 0 0 1 1 0] [0 0 1 0 0 0 1 1 0] 

[1 0 0 0 1 1 1 0 0] [1 0 0 1 1 0 0 1 0] 
[0 1 0 0 0 1 0 0 1] [0 1 0 0 1 0 0 0 1] 
[0 0 1 0 0 0 1 1 0] [0 0 1 0 0 1 1 1 0] 

[1 0 0 0 1 1 0 0 0] [1 0 0 1 1 0 0 1 0] 
[0 1 0 0 1 0 1 0 0] [0 1 0 0 0 1 1 1 0] 
[0 0 1 0 1 0 0 0 1] [0 0 1 0 0 1 0 0 1] 

[1 0 0 1 0 1 1 0 0] [1 0 0 1 0 1 0 1 1] 
[0 1 0 0 0 1 0 1 0] [0 1 0 1 0 0 1 0 0] 
[0 0 1 0 0 0 1 0 1] [0 0 1 0 1 0 0 0 0] 

[1 0 0 1 0 0 1 0 1] [1 0 0 1 1 0 0 0 1] 
[0 1 0 1 0 0 0 1 0] [0 1 0 1 0 1 1 0 0] 
[0 0 1 0 1 1 1 0 0] [0 0 1 0 1 0 0 1 0] 

[1 0 0 1 1 0 0 0 1] [1 0 0 0 1 1 0 1 0] 
[0 1 0 1 0 0 1 0 0] [0 1 0 0 1 0 1 0 0] 
[0 0 1 0 0 1 0 1 1] [0 0 1 0 0 1 0 0 1] 

[1 0 0 1 0 0 1 0 0] [1 0 0 1 1 0 1 0 0] 
[0 1 0 1 0 0 0 1 1] [0 1 0 0 0 1 0 0 0] 
[0 0 1 0 1 1 0 0 1] [0 0 1 0 0 0 0 1 0] 

[1 0 0 1 1 0 0 0 0] [1 0 0 1 1 0 0 0 0] 
[0 1 0 1 0 0 0 1 0] [0 1 0 1 0 0 0 0 1] 
[0 0 1 0 0 1 0 0 0] [0 0 1 0 0 0 0 1 0] 

[1 0 0 0 1 1 0 0 1] [1 0 0 1 1 1 0 1 0] 
[0 1 0 0 1 0 1 0 0] [0 1 0 0 0 0 0 1 1] 
[0 0 1 0 0 1 0 1 0] [0 0 1 0 0 0 0 0 0] 

[1 0 0 1 0 1 0 0 1] [1 0 0 1 0 1 0 0 1] 
[0 1 0 1 0 0 1 0 0] [0 1 0 0 1 0 0 1 1] 
[0 0 1 1 0 0 0 1 0] [0 0 1 0 0 0 1 0 0] 

[1 0 0 1 0 1 0 1 0] [1 0 0 1 1 1 0 0 0] 
[0 1 0 0 0 1 0 0 1] [0 1 0 0 0 1 0 0 1] 
[0 0 1 0 0 0 0 0 0] [0 0 1 0 0 0 1 0 0] 

[1 0 0 1 0 1 0 0 1] [1 0 0 1 1 1 0 0 0] 
[0 1 0 0 1 0 1 0 0] [0 1 0 0 0 0 1 0 1] 
[0 0 1 0 0 0 0 0 0] [0 0 1 0 0 0 0 1 0] 

[1 0 0 1 1 0 0 0 0] [1 0 0 1 0 1 0 0 0] 
[0 1 0 1 0 1 0 1 1] [0 1 0 1 0 0 1 0 0] 
[0 0 1 1 0 0 1 0 0] [0 0 1 0 0 0 0 1 0] 

[1 0 0 1 0 1 0 0 0] [1 0 0 0 1 0 0 1 1] 
[0 1 0 1 0 0 0 0 1] [0 1 0 0 0 1 1 1 0] 
[0 0 1 0 1 0 0 0 0] [0 0 1 0 0 0 0 0 0] 

[1 0 0 1 1 0 0 0 0] [1 0 0 1 1 0 1 0 0] 
[0 1 0 1 0 0 0 1 1] [0 1 0 1 0 1 0 0 1] 
[0 0 1 0 0 1 1 1 0] [0 0 1 0 1 0 0 1 1] 

[1 0 0 1 0 0 1 0 1] [1 0 0 1 0 0 1 0 0] 
[0 1 0 0 1 0 0 0 0] [0 1 0 1 0 0 0 0 1] 
[0 0 1 0 0 1 0 0 0] [0 0 1 0 0 0 0 1 0] 

[1 0 0 1 0 0 1 0 0] [1 0 0 1 1 0 0 1 0] 
[0 1 0 0 1 0 0 0 0] [0 1 0 1 0 1 0 0 0] 
[0 0 1 0 0 1 0 1 0] [0 0 1 0 0 0 1 0 0] 

[1 0 0 1 1 0 0 1 0] [1 0 0 1 0 1 0 0 0] 
[0 1 0 0 1 0 1 0 1] [0 1 0 0 1 0 1 1 0] 
[0 0 1 0 0 0 0 0 0] [0 0 1 0 0 0 0 1 1] 

[1 0 0 1 0 0 0 1 0] [1 0 0 1 0 0 0 1 0] 
[0 1 0 0 1 0 0 0 0] [0 1 0 0 0 1 0 0 0] 
[0 0 1 0 0 1 0 0 1] [0 0 1 0 0 0 1 0 1] 

[1 0 0 1 1 0 0 0 0] [1 0 0 1 1 0 0 0 0] 
[0 1 0 1 0 1 0 0 0] [0 1 0 1 0 0 1 0 0] 
[0 0 1 1 0 0 0 0 1] [0 0 1 1 0 0 0 0 1] 

[1 0 0 0 1 0 0 0 0] [1 0 0 1 1 0 1 0 0] 
[0 1 0 0 0 1 1 0 1] [0 1 0 0 1 0 0 1 0] 
[0 0 1 0 0 0 1 1 0] [0 0 1 0 0 1 0 0 0] 

[1 0 0 1 1 0 0 0 0] [1 0 0 1 1 1 0 0 0] 
[0 1 0 0 0 0 1 1 0] [0 1 0 1 0 0 1 0 1] 
[0 0 1 0 0 0 1 0 1] [0 0 1 0 1 0 1 1 0] 

[1 0 0 1 0 0 0 0 0] [1 0 0 1 0 0 0 0 0] 
[0 1 0 0 1 0 0 1 0] [0 1 0 0 0 1 1 0 0] 
[0 0 1 0 1 0 0 0 1] [0 0 1 0 0 1 0 0 1] 

[1 0 0 1 1 1 1 0 0] [1 0 0 1 0 1 1 1 0] 
[0 1 0 1 0 0 1 1 1] [0 1 0 1 0 0 0 0 1] 
[0 0 1 0 0 0 0 0 0] [0 0 1 0 0 0 0 0 0] 

[1 0 0 1 1 1 1 0 0] [1 0 0 1 1 1 1 0 0] 
[0 1 0 1 0 0 0 1 0] [0 1 0 1 0 0 0 0 1] 
[0 0 1 0 1 0 0 0 1] [0 0 1 0 0 1 0 1 0] 

[1 0 0 0 1 1 1 0 0] [1 0 0 1 0 1 1 0 0] 
[0 1 0 0 0 0 1 0 1] [0 1 0 0 0 1 0 0 1] 
[0 0 1 0 0 0 0 0 0] [0 0 1 0 0 0 0 1 0] 

[1 0 0 1 0 1 1 0 0] [1 0 0 0 1 1 0 0 1] 
[0 1 0 0 0 0 1 1 1] [0 1 0 0 0 1 1 0 0] 
[0 0 1 0 0 0 0 0 0] [0 0 1 0 0 0 0 0 0] 

[1 0 0 1 0 0 1 0 1] [1 0 0 1 1 1 1 0 1] 
[0 1 0 0 1 1 1 0 0] [0 1 0 0 0 1 0 1 0] 
[0 0 1 0 0 0 0 1 0] [0 0 1 0 0 0 0 0 0] 

[1 0 0 1 0 0 0 1 1] [1 0 0 1 1 0 1 0 0] 
[0 1 0 0 1 1 1 0 1] [0 1 0 0 1 1 0 0 1] 
[0 0 1 0 0 0 0 0 0] [0 0 1 0 0 1 0 1 0] 

[1 0 0 0 1 0 1 0 1] [1 0 0 0 1 0 1 0 0] 
[0 1 0 0 1 0 0 1 0] [0 1 0 0 1 0 0 0 1] 
[0 0 1 0 0 0 0 0 0] [0 0 1 0 0 1 0 0 0] 

[1 0 0 1 1 1 0 1 0] [1 0 0 1 0 1 0 0 0] 
[0 1 0 0 1 0 1 0 0] [0 1 0 1 0 0 0 1 0] 
[0 0 1 0 0 0 0 1 1] [0 0 1 1 0 0 0 0 1] 

[1 0 0 1 1 1 0 0 1] [1 0 0 1 1 1 0 0 1] 
[0 1 0 0 0 1 1 0 0] [0 1 0 1 0 0 1 0 0] 
[0 0 1 0 0 0 0 1 0] [0 0 1 1 0 0 0 1 0] 

[1 0 0 1 1 1 0 0 1] [1 0 0 1 1 1 0 0 0] 
[0 1 0 1 0 0 0 1 0] [0 1 0 1 0 0 1 0 0] 
[0 0 1 0 0 1 1 0 0] [0 0 1 0 0 1 0 1 0] 

[1 0 0 1 1 1 0 0 0] [1 0 0 1 1 1 0 0 0] 
[0 1 0 1 0 0 0 1 0] [0 1 0 1 0 0 0 0 1] 
[0 0 1 0 1 0 1 0 0] [0 0 1 0 1 0 1 0 0] 

[1 0 0 0 1 0 0 0 1] [1 0 0 1 1 1 0 0 0] 
[0 1 0 0 0 1 0 0 0] [0 1 0 0 1 0 1 0 0] 
[0 0 1 0 0 0 1 1 0] [0 0 1 0 1 0 0 1 0] 

[1 0 0 1 1 1 0 0 0] [1 0 0 1 1 1 0 0 0] 
[0 1 0 0 1 0 0 1 0] [0 1 0 0 1 0 0 0 1] 
[0 0 1 0 0 1 1 0 0] [0 0 1 0 0 0 1 1 0] 

[1 0 0 1 1 1 0 0 0] [1 0 0 1 1 0 1 0 0] 
[0 1 0 0 0 1 1 1 0] [0 1 0 1 0 0 0 1 1] 
[0 0 1 0 0 0 0 0 1] [0 0 1 0 1 1 0 1 0] 

[1 0 0 1 0 0 1 1 0] [1 0 0 1 0 0 1 0 1] 
[0 1 0 0 1 0 0 0 0] [0 1 0 0 0 1 0 1 1] 
[0 0 1 0 0 0 0 1 1] [0 0 1 0 0 0 0 0 0] 

[1 0 0 0 1 0 0 0 0] [1 0 0 1 1 0 0 1 0] 
[0 1 0 0 0 1 0 1 0] [0 1 0 0 0 1 0 0 0] 
[0 0 1 0 0 1 0 0 1] [0 0 1 0 0 0 0 1 1] 
\end{lstlisting}

\clearpage   

        % Use IEEE DIN 1505 style for bibliography / Literaturverzeichnisses
        \bibliographystyle{ieeetransa}
        %\nocite{*}              % Include all references without checking / Alle References immer aufführen
        \bibliography{./refs/final_ref_bib}

\end{document}

%
% EOF!
%