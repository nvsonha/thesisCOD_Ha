%%%%%%%%%%%%%%%%%%%%%%%%%%%%%%%%%%%%%%%%%%%%%%%
\chapter{Introduction} \label{chap:introduction}
%%%%%%%%%%%%%%%%%%%%%%%%%%%%%%%%%%%%%%%%%%%%%%%

\textit{Network coding} was first introduced in Ahlswede et al.'s
seminal paper \cite{Ahlswede:2000}. Network coding gives a huge advantage
in increasing throughput of information transmission in comparison
with simple routing methods for a communication network. This throughput
gain is achieved in network coding, since nodes in a network are allowed
to forward a \textit{function} of their received packets, while in
routing such packets can only be forwarded to another node. K\"otter
and M\'edard provided an algebraic formulation for a \textit{linear
network coding} problem and its scalar solvability. If functions on
all nodes are linear, then we obtain a linear network coding solution,
and a \textit{solution} is an assignment of all functions such that
all destination nodes can recover all of their requested messages
transmitted from a source node. Network coding was further developed
with \textit{vector network coding} by Ebrahimi \cite{Ebrahimi:2011},
where all packets are vectors of length $t$. In \cite{Wachter-Zeh:2018},
Etzion and Wachter-Zeh proved that vector network coding based on
subspace codes outperforms linear network coding. It gives the motivation
for our study in this thesis on vector network coding, especially
the study of \textit{gap} measuring the difference in \textit{alphabet
sizes} between solutions of scalar and vector network coding. The
alphabet size is an important parameter determining the amount of
computation performed at each network node \cite{Wachter-Zeh:2018}.
The problem of finding the minimum required alphabet size of a (linear
or nonlinear) scalar network code for a certain multicast network
is NP-complete \cite{Langberg:2009,Lehman:2004,Gone:2018}. 

\textit{Generalized combination networks} (GCN) were first introduced
in \cite{Wachter-Zeh:2018}, where they have shown many instances
of the network with vector solutions outperforming scalar solutions.
However, no general vector solution was found a generalized combination
network with 3 source messages. It motivates us to study it in this
study, together with the gap for such a solution. We further develope
the study on gap for 2 families of GCN by both combinatorial approaches
and computational approaches. 

\textbf{Outline}

In \textbf{Chapter 2}, we first recall coding-theory-specific notions
and give an introduction to the known codes that we consider in this
thesis. Then, we give the definition of \textit{maximum rank distance}
(MRD) code and its properties. This code was mainly used to study
vector solutions for multiple families of the \textit{generalized
combination network} (GCN) in \cite{Wachter-Zeh:2018}. Finally, we
give the definition of Grassmannian code, Covering Grassmannian code,
Multiple Grassmannian code and the notion of the maximum size of a
Multiple Grammanian code. Because Grassmannian codes contain subspaces
of the same dimension over a finite field $\ensuremath{\mathbb{F}}_{q}$,
they have been recently applied in the study of network coding problems,
such as \cite{Etzion:2016,Etzion:2018,Wachter-Zeh:2018,Zhang:2019}.

The remaining chapters contain our study of new gap sizes for GCN.
We divided them into three main parts: Chapter 3 and 4 explain how
we transfer vector network coding problems for GCN to problems of
finding matrices or Grassmanian codes, Chapter 5 contains new gap
sizes for 2 families of GCN with combinatorial proofs based on Lov\'asz
Local Lemma (LLL), and Chapter 6 contains new computational results
of vector solutions outperforming scalar solutions for the $\left(\epsilon=1,\ell=1\right)-\mathcal{N}_{h=3,r,s=4}$
network. The details of each chapter are mentioned below.

In \textbf{Chapter 3} and \textbf{Chapter 4}, we represent network
as a matrix channel and introduce an advantage of vector solutions
in alphabet sizes in comparison with scalar solutions for network
coding problems. We firstly recall the motivation of network coding,
and secondly we introduce how we approach such problems by fomulating
the relationship between source's messages and receiver's packets
by linear equation systems. Thirdly, we explain why we choose GCN
for our study, and we recall separately known theorems on an existence
of a scalar or a vector solution for GCN. Finally, we formulate the
\textit{gap} to measure the difference in alphabet sizes between a
vector solution and a corresponding optimal scalar solution. We list
known gaps for some instances of GCN in previous studies together
with our new found gaps in this study.

In \textbf{Chapter 5}, we present new gaps found by an approach based
on LLL. We start this chapter with a simple network $\left(\epsilon=1,\ell=1\right)-\mathcal{N}_{h=3,r,s=4}$.
No general vector solution outperforming scalar network coding was
found in previous studies. The gap of the $\left(\epsilon=1,\ell=1\right)-\mathcal{N}_{h=3,r,s=4}$
network for such purpose is found with combinatorial proofs in this
study. During the proof of the gap, we also prove there always exists
vector solutions for the network, if and only if the number $r$ of
intermediate nodes is less than or equal to a certain number. After
achieving the gap for the $\left(\epsilon=1,\ell=1\right)-\mathcal{N}_{h=3,r,s=4}$
network, we develop the proofs further for the $\left(\epsilon=1,\ell=1\right)-\mathcal{N}_{h,r,s}$
network and the $\left(\epsilon=1,\ell>1\right)-\mathcal{N}_{2\ell,r,2\ell+1}$
network. Knowing the gap motivates us to search for vector solutions
achieving such gap, which leads to computational results presented
in Chapter 6.

\textbf{Chapter 6} shows the core steps of 4 different computational
approaches to find a vector solution outperforming the optimal scalar
solution for the $\left(\epsilon=1,\ell=1\right)-\mathcal{N}_{h=3,r,s=4}$
with $t=2$ and $t=3$. We have found a vector solution of 89 nodes
which is roughly two times 42 nodes of the optimal scalar solution
for the network with $t=2$. For the network with $t=3$, a vector
solution of 166 nodes is found when the optimal scalar solution of
such network exists if and only the number of nodes is less than or
equal to 146. We then conclude the new bound on maximum size of Grassmanian
codes for such network respectively, $89\leq\mathcal{A}_{2}\left(6,4,3;2\right)\leq126$
and $166\leq\mathcal{A}_{2}\left(9,6,3;2\right)\leq537$. While writing
this thesis, the bound of the code when $t=2$ has been improved in
\cite{Etzion:2018}. The result of $t=3$ have not yet been found
in our scope of knowledge. 

\clearpage
