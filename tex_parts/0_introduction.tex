%%%%%%%%%%%%%%%%%%%%%%%%%%%%%%%%%%%%%%%%%%%%%%%
\chapter{Introduction} \label{chap:introduction}
%%%%%%%%%%%%%%%%%%%%%%%%%%%%%%%%%%%%%%%%%%%%%%%

For the traditional way of network coding, scalar solutions are used. However, it was
proved in Professor Antonia's paper that vector solutions outperform scalar solutions
in specic cases. It means that we can connect more devices into our network, which
is especially meaningful for the IOT devices. The overview of networks where vector
solutions outperform the scalar solution is described in Chapter X (Not Yet Inserted).
Then we consider a case that is unable to solve by subspace codes or rank-metric codes
in Chapter 2. Our computational method shows better results than the scalar solution,
67 results and 166 results respectively in case of t=2 and t=3, where scalar solution has
only 42 results and 146 results. The method is described in Chapter 3.


\clearpage
