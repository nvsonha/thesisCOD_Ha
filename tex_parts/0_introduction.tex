%%%%%%%%%%%%%%%%%%%%%%%%%%%%%%%%%%%%%%%%%%%%%%%
\chapter{Introduction} \label{chap:introduction}
%%%%%%%%%%%%%%%%%%%%%%%%%%%%%%%%%%%%%%%%%%%%%%%

\textit{Network coding} was first introduced in Ahlswede et al.'s
seminal paper \cite{Ahlswede:2000}. Network coding gives an advantage
in increasing throughput of information transmission in comparison
with simple routing methods for a communication network \cite{Li:2003,Ho:2003}.
The considered communication network was a directed graph with nodes
connected with each other by multiple links. The throughput gain is
achieved in network coding, since the nodes are allowed to forward
a \textit{function} of their received packets, while in routing such
packets can only be forwarded to another node. K\"otter and M\'edard
provided an algebraic formulation for a \textit{linear network coding}
problem and its scalar solvability. If functions of the packets on
the links of the network, then we obtain a linear network coding solution,
and a \textit{solution} is an assignment of these functions such that
all destination nodes can recover all of their requested messages
transmitted from a source node. Network coding was further developed
with \textit{vector network coding} by Ebrahimi \cite{Ebrahimi:2011},
where all packets are vectors of length $t$. In \cite{Wachter-Zeh:2018},
Etzion and Wachter-Zeh proved that vector network coding based on
subspace codes outperforms linear network coding for several generalizations
of the well-known combination networks \cite{Riis:2006}. It gives
the motivation for our study in this thesis on vector network coding,
especially the study of \textit{gap} measuring the difference in \textit{alphabet
sizes} between solutions of scalar and vector network coding. The
alphabet size is an important parameter determining the amount of
computation performed at each node \cite{Wachter-Zeh:2018}. In this
thesis, we show that smaller alphabet sizes can be achieved by vector
network coding for further instances of the \textit{generalized combination
networks} (GCN) \cite{Wachter-Zeh:2018}, which allows higher number
of destination nodes to be connected to the network in comparison
with scalar network coding.

\textbf{Outline}

In \textbf{Chapter 2}, we recall coding-theory-specific notions and
give an introduction to the known codes that we consider in this thesis.
We first give the definition of \textit{maximum rank distance} (MRD)
code and its properties. This code was mainly used to study vector
solutions for several families of the GCN in \cite{Wachter-Zeh:2018}.
Then, we give the definition of Grassmannian code, Covering Grassmannian
code, Multiple Grassmannian code and the notion of the maximum size
of a Multiple Grammanian code. Since Grassmannian codes contain subspaces
of the same dimension over a finite field $\ensuremath{\mathbb{F}}_{q}$,
they have been recently applied in the study of network coding problems,
such as \cite{Etzion:2016,Etzion:2018,Wachter-Zeh:2018,Zhang:2019}.

In \textbf{Chapter 3} and \textbf{Chapter 4}, we represent networks
as matrix channels and introduce how vector solutions outperform scalar
solutions in alphabet sizes for network coding problems. We firstly
recall the motivation of network coding, and secondly we explain our
approach by fomulating the relationship between source's messages
and receiver's packets by linear equation systems. Thirdly, we explain
why we choose GCN for our study, and we recall known bounds on alphabet
size between scalar and vector solutions for GCN. Finally, we formulate
the \textit{gap} to measure the difference in alphabet sizes between
a vector solution and a corresponding optimal scalar solution. We
list known gaps for some instances of GCN in previous studies together
with our new found gaps in this study.

The remaining chapters contain new results, i.e. our study of new
gap sizes for GCN. We divided them into two main parts: Chapter 5
contains new gap sizes for three families of GCN with combinatorial
proofs based on Lov\'asz Local Lemma (LLL), and Chapter 6 and 7 contains
new computational results of vector solutions outperforming scalar
solutions for the $\left(\epsilon=1,\ell=1\right)-\mathcal{N}_{h=3,r,s=4}$
network. The details of each chapter are mentioned below.

In \textbf{Chapter 5}, we present new gaps found by combinatorial
approaches based on LLL. We begin this chapter with a simple network,
namely the $\left(\epsilon=1,\ell=1\right)-\mathcal{N}_{h=3,r,s=4}$
network, and the gap of this network is first found in our study.
We prove that there exists vector solutions for the network, if and
only if the number $r$ of intermediate nodes is less than or equal
to a certain number. After achieving the gap for the $\left(\epsilon=1,\ell=1\right)-\mathcal{N}_{h=3,r,s=4}$
network, we develop the proofs further for the $\left(\epsilon=1,\ell=1\right)-\mathcal{N}_{h,r,s}$
network, the $\left(\epsilon>1,\ell=1\right)-\mathcal{N}_{h,r,s}$
network and the $\left(\epsilon=1,\ell>1\right)-\mathcal{N}_{2\ell,r,2\ell+1}$
network. Knowing the gap motivates us to search for vector solutions
achieving such gap, which leads to computational results presented
in Chapter 6.

\textbf{Chapter 6} shows the core steps of 4 different computational
approaches to find vector solutions outperforming the optimal scalar
solutions for the $\left(\epsilon=1,\ell=1\right)-\mathcal{N}_{h=3,r,s=4}$
with $t=2$ and $t=3$. We have found vector solutions of 89 nodes
and 166 nodes, while the scalar solution of such network exists if
and only if $r\leq42$ and $r\leq146$ respectively. We then conclude
the new bound on maximum size of Grassmanian codes for the network,
$89\leq\mathcal{A}_{2}\left(6,4,3;2\right)\leq126$ and $166\leq\mathcal{A}_{2}\left(9,6,3;2\right)\leq537$.
While writing this thesis, the bound of $\mathcal{A}_{2}\left(6,4,3;2\right)$
has been improved in \cite{Etzion:2018} and the result of $t=3$
have not yet been found in any other studies. 

The thesis is concluded in \textbf{Chapter 7}. 

\clearpage
