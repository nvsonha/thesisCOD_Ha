%%%%%%%%%%%%%%%%%%%%%%%%%%%%%%%%%%%%%%%%%%%%%%%
\chapter{Preliminaries} \label{chap:preliminaries}
%%%%%%%%%%%%%%%%%%%%%%%%%%%%%%%%%%%%%%%%%%%%%%%

\section{Notation and Basic Terminology}

\paragraph{Vectors and Matrices}

Vectors $\boldsymbol{v}$ are denoted by underlined letters. Unless
stated otherwise, vectors are indexed starting from 1, i.e. $\boldsymbol{v}=\left[v_{1},\ldots,v_{n}\right]$.
Vectors are usually considered to be row vectors. Matrices $\boldsymbol{V}$
are shown in bold and capital letters. Elements of matrices or vectors
are surrounded by square brackets, and elements of tuples are surrounded
by round brackets.

\paragraph{Vector space}

A vector space of dimension $n$ over a finite field with $q$ elements
is denoted by $\ensuremath{\mathbb{F}}_{q}^{n}$. 

\paragraph{Linear Block Code}

>\textcompwordmark >\textcompwordmark > WHY ONLY LINEAR?

\textit{Block codes} are finite length codes first studied by Golay
\cite{Golay:1949} and Hamming \cite{Hamming:1950}. A \textit{linear}
$\left[n,k\right]_{q}$ \textit{block code} $\ensuremath{\mathcal{C}}$
of length $n$, dimension $k$ and codimension (or redundancy) $r=n-k$,
is a $k$-dimensional \textit{linear subspace} of $\ensuremath{\mathbb{F}}_{q}^{n}$.
A \textit{generator matrix} $\boldsymbol{G}$ has $k$ rows, which
form a \textit{basis} of $\ensuremath{\mathcal{C}}$, i.e., $\boldsymbol{G}$'s
row space is a set of $k$ linearly independent vectors generating
$\ensuremath{\mathcal{C}}$. The codeword $\underline{c}\in\ensuremath{\mathcal{C}}$
is encoded by the \textit{information vector} (or \textit{message})
$\boldsymbol{m}$, with $\boldsymbol{c}=\boldsymbol{m}\cdot\boldsymbol{G}$.
For any $\boldsymbol{c}\in\ensuremath{\mathcal{C}}$: $\boldsymbol{c}\cdot\boldsymbol{H}^{T}=\boldsymbol{0}$,
with $\boldsymbol{H}$ is a \textit{parity-check matrix} of $\ensuremath{\mathcal{C}}$,
whose row space generates the $\left[n,n-k\right]_{q}$ dual code
$\ensuremath{\mathcal{C}}^{\perp}$. $\left[n,k,d\right]_{q}$ is
equivalent to $\left[n,k\right]_{q}$ with minimum distance $d$ fulfills
$d=\underset{\boldsymbol{c}\in\ensuremath{\mathcal{C}\setminus\left\{ \boldsymbol{0}\right\} }}{min}d\left(\boldsymbol{c},\boldsymbol{0}\right)$
with respect to a metric $d\left(\cdot,\cdot\right):\ensuremath{\mathbb{F}}^{n}\times\ensuremath{\mathbb{F}}^{n}\rightarrow\mathbb{R}_{\geq0}$
on $\ensuremath{\mathbb{F}}^{n}$ \footnote{i.e., $d\left(\boldsymbol{x},\boldsymbol{y}\right)\geq0$,$d\left(\boldsymbol{x},\boldsymbol{y}\right)=0$
iff $\boldsymbol{x}=\boldsymbol{y}$, $d\left(\boldsymbol{x},\boldsymbol{y}\right)=d\left(\boldsymbol{y},\boldsymbol{x}\right)$
and $d\left(\boldsymbol{x},\boldsymbol{z}\right)\leq d\left(\boldsymbol{x},\boldsymbol{y}\right)+d\left(\boldsymbol{y},\boldsymbol{z}\right)$
for all $\boldsymbol{x},\boldsymbol{y},\boldsymbol{z}\in\ensuremath{\mathbb{F}}^{n}$.}.

\paragraph{Gaussian coefficient}

Gaussian coefficient (also known as $q$-binomial) counts the number
of subspaces of dimension $k$ in a vector space $\ensuremath{\mathbb{F}}_{q}^{n}$,

\[
\left[\begin{array}{c}
n\\
k
\end{array}\right]_{q}=\stackrel[i=0]{k-1}{\prod}\frac{q^{n}-q^{i}}{q^{k}-q^{i}}
\]


\paragraph{Multigraph}

A graph is permitted to have multiple edges. Edges that are incident
to same vertices can be in parallel. 

\paragraph{Directed Acyclic Graph}

A finite directed graph with no directed cycles, i.e. it consists
of a finite number vertices and edges, with each edge directed from
a vertex to another, such that there is no loop from any vertex $v$
with a sequence of directed edges back to the vertex again $v$.

\paragraph{Multicast}

Multicast communication supports the distribution of a data packet
to a group of users \cite{Zhang:2012}. It can be one-to-many or many-to-many
distribution \cite{Harte:2008}. In this study, we consider only one-to-many
multicast network.

\paragraph{Asymptotic Behavior}

For the combinatorial results, we study the asymtotic behaviour of
some formulas depending on the alphabet size $q$ and the vector length
$t$, by using the Bachmann-Landau notation, i.e. $\mathcal{O}\left(f\left(q,t\right)\right)$
for upper, $\Theta\left(f\left(q,t\right)\right)$ for tight, and
$\Omega\left(f\left(q,t\right)\right)$ for lower bounds, where $f$
is a function of the alphabet size and the vector length.

\section{Definition}

>\textcompwordmark >\textcompwordmark > EXPLAIN WHY I NEED EACH
DEFINITION?
\begin{defn}[Grassmannian Code]
 A Grassmannian code is a set of all subspaces of dimension $k\leq n$
in $\ensuremath{\mathbb{F}}_{q}^{n}$, and is denoted by $\mathcal{G}_{q}\left(n,k\right)$.
Due to being the set of all subspaces that have the same dimension
$k$, it is also called a \textit{constant dimension code}. \cite{Zhang:2019}
\end{defn}
%
\begin{defn}[Projective Space]
 The \textit{projective space of order} $n$ is a set of all subspaces
of $\ensuremath{\mathbb{F}}_{q}^{n}$, and is denoted by $\mathcal{P}_{q}\left(n\right)$,
i.e. a union of all dimension $k=0,\ldots n$ subspaces in $\ensuremath{\mathbb{F}}_{q}^{n}$
or $\mathcal{P}_{q}\left(n\right)=\bigcup_{k=0}^{n}\mathcal{G}_{q}\left(n,k\right)$.
\cite{Wachter-Zeh:2018}
\end{defn}
%
\begin{defn}[Covering Grassmannian code]
 An $\alpha-\left(n,k,\delta\right)_{q}^{c}$ covering Grassmannian
code (code in short) $\mathcal{C}$ is a subset of $\mathcal{G}_{q}\left(n,k\right)$
such that each subset of $\alpha$ codewords of $\mathcal{C}$ span
a subspace whose dimension is at least $\delta+k$ in $\ensuremath{\mathbb{F}}_{q}^{n}$.
\cite{Zhang:2019}
\end{defn}

\paragraph{The cardinality of a Grassmannian code}

The cardinality of $\mathcal{G}_{q}\left(n,k\right)$ is the Gaussian
coefficient (also known as $q$-binomial), which counts the number
of subspaces of dimension $k$ in a vector space $\ensuremath{\mathbb{F}}_{q}^{n}$,

\[
\left|\mathcal{G}_{q}\left(n,k\right)\right|=\left[\begin{array}{c}
n\\
k
\end{array}\right]_{q}=\stackrel[i=0]{k-1}{\prod}\frac{q^{n}-q^{i}}{q^{k}-q^{i}},
\]

where $q^{\left(n-k\right)k}\leq\left[\begin{array}{c}
n\\
k
\end{array}\right]_{q}\leq4q^{\left(n-k\right)k}$.

\begin{defn}[Subspace packing \cite{Etzion:2018}]
 A subspace packing $t-\left(n,k,\lambda\right)_{q}^{m}$ is a set
$\mathcal{S}$ of $k$-subspaces or $k$-dimensional subspaces (called
\textit{blocks}), such that each $t$-subspace of $\ensuremath{\mathbb{F}}_{q}^{n}$
is contained in at most $\lambda$ codewords of $\mathcal{C}$. 
\end{defn}
%
\begin{defn}[\cite{Etzion:2018}]
$\mathcal{A}_{q}\left(n,k,t;\lambda\right)$ denotes the maximum
size of a $t-\left(n,k,\lambda\right)_{q}^{m}$ code, where there
are no repeated codewords. 
\end{defn}

\section{Theorem}

>\textcompwordmark >\textcompwordmark > EXPLAIN BEFORE WHY I NEED
EACH THEOREM?
\begin{thm}[\cite{Zhang:2019}]
 If $n,k,t,$ and $\lambda$ are positive integers such that $1\leq t<k<n$
and $1\leq\lambda\leq\left[\begin{array}{c}
n-t\\
k-t
\end{array}\right]_{q}$, then

\[
\mathcal{A}_{q}\left(n,k,t;\lambda\right)\leq\left\lfloor \lambda\frac{\left[\begin{array}{c}
n\\
t
\end{array}\right]_{q}}{\left[\begin{array}{c}
k\\
t
\end{array}\right]_{q}}\right\rfloor 
\]
\end{thm}
%
\begin{thm}[\cite{Zhang:2019}]
 If $n,k,t,$ and $\lambda$ are positive integers such that $1\leq t<k<n$
and $1\leq\lambda\leq\left[\begin{array}{c}
n-t\\
k-t
\end{array}\right]_{q}$, then

\[
\mathcal{A}_{q}\left(n,k,t;\lambda\right)\leq\left\lfloor \frac{q^{n}-1}{q^{k}-1}\mathcal{A}_{q}\left(n-1,k-1,t-1;\lambda\right)\right\rfloor 
\]
\end{thm}

\clearpage