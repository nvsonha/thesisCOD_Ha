%%%%%%%%%%%%%%%%%%%%%%%%%%%%%%%%%%%%%%%%%%%%%%%
\chapter{Preliminaries} \label{chap:preliminaries}
%%%%%%%%%%%%%%%%%%%%%%%%%%%%%%%%%%%%%%%%%%%%%%%

\begin{defn}[Grassmannian Code]
 A Grassmannian code is a set of all subspaces of dimension $k\leq n$
in $\ensuremath{\mathbb{F}}_{q}^{n}$, and is denoted by $\mathcal{G}_{q}\left(n,k\right)$.
Due to being the set of all subspaces that have the same dimension
$k$, it is also called a \textit{constant dimension code}. The cardinality
of $\mathcal{G}_{q}\left(n,k\right)$ is the Gaussian coefficient
(also known as $q$-binomial), which counts the number of subspaces
of dimension $k$ in a vector space $\ensuremath{\mathbb{F}}_{q}^{n}$:

\[
\left|\mathcal{G}_{q}\left(n,k\right)\right|=\left[\begin{array}{c}
n\\
k
\end{array}\right]_{q}=\stackrel[i=0]{k-1}{\prod}\frac{q^{n}-q^{i}}{q^{k}-q^{i}},
\]

where $q^{\left(n-k\right)k}\leq\left[\begin{array}{c}
n\\
k
\end{array}\right]_{q}\leq4q^{\left(n-k\right)k}$.
\end{defn}

\clearpage