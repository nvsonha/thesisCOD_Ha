%%%%%%%%%%%%%%%%%%%%%%%%%%%%%%%%%%%%%%%%%%%%%%%
\chapter{Preliminaries} \label{chap:preliminaries}
%%%%%%%%%%%%%%%%%%%%%%%%%%%%%%%%%%%%%%%%%%%%%%%

\section{Definition}
\begin{defn}[Vector Space]
 A vector space of dimension $n$ over a finite field with $q$ elements
is denoted by $\ensuremath{\mathbb{F}}_{q}^{n}$.
\end{defn}
%
\begin{defn}[Gaussian coefficient]
 Gaussian coefficient (also known as $q$-binomial), which counts
the number of subspaces of dimension $k$ in a vector space $\ensuremath{\mathbb{F}}_{q}^{n}$:

\[
\left[\begin{array}{c}
n\\
k
\end{array}\right]_{q}=\stackrel[i=0]{k-1}{\prod}\frac{q^{n}-q^{i}}{q^{k}-q^{i}}
\]
\end{defn}
%
\begin{defn}[Grassmannian Code]
 A Grassmannian code is a set of all subspaces of dimension $k\leq n$
in $\ensuremath{\mathbb{F}}_{q}^{n}$, and is denoted by $\mathcal{G}_{q}\left(n,k\right)$.
Due to being the set of all subspaces that have the same dimension
$k$, it is also called a \textit{constant dimension code}. The cardinality
of $\mathcal{G}_{q}\left(n,k\right)$ is the Gaussian coefficient
(also known as $q$-binomial), which counts the number of subspaces
of dimension $k$ in a vector space $\ensuremath{\mathbb{F}}_{q}^{n}$:

\[
\left|\mathcal{G}_{q}\left(n,k\right)\right|=\left[\begin{array}{c}
n\\
k
\end{array}\right]_{q}=\stackrel[i=0]{k-1}{\prod}\frac{q^{n}-q^{i}}{q^{k}-q^{i}},
\]

where $q^{\left(n-k\right)k}\leq\left[\begin{array}{c}
n\\
k
\end{array}\right]_{q}\leq4q^{\left(n-k\right)k}$.
\end{defn}
%
\begin{defn}[Projective Space]
 The \textit{projective space of order} $n$ is a set of all subspaces
of $\ensuremath{\mathbb{F}}_{q}^{n}$, and is denoted by $\mathcal{P}_{q}\left(n\right)$,
i.e. a union of all dimension $k=0,\ldots n$ subspaces in $\ensuremath{\mathbb{F}}_{q}^{n}$
or $\mathcal{P}_{q}\left(n\right)=\bigcup_{k=0}^{n}\mathcal{G}_{q}\left(n,k\right)$.
\end{defn}
%
\begin{defn}[Covering Grassmannian code]
 An $\alpha-\left(n,k,\delta\right)_{q}^{c}$ covering Grassmannian
code (code in short) $\mathcal{C}$ is a subset of $\mathcal{G}_{q}\left(n,k\right)$
such that each subset of $\alpha$ codewords of $\mathcal{C}$ span
a subspace whose dimension is at least $\delta+k$ in $\ensuremath{\mathbb{F}}_{q}^{n}$.
\end{defn}
%
\begin{defn}[Multiple Grassmannian code]
 A $t-\left(n,k,\lambda\right)_{q}^{m}$ multiple Grassmannian code
(code in short) $\mathcal{C}$ is a subset of $\mathcal{G}_{q}\left(n,k\right)$
such that each $t$-subspace of $\ensuremath{\mathbb{F}}_{q}^{n}$
is contained in at most $\lambda$ codewords of $\mathcal{C}$. Following
to \cite{Zhang:2019}, $m$ refers to multiplicity.
\end{defn}
%
\begin{defn}
$\mathcal{A}_{q}\left(n,k,t;\lambda\right)$ denotes the maximum size
of a $t-\left(n,k,\lambda\right)_{q}^{m}$ code, where there are no
repeated codewords.
\end{defn}
%
\begin{defn}[Subspace packing]
 A subspace packing $t-\left(n,k,\lambda\right)_{q}^{m}$ is a set
$\mathcal{S}$ of $k$-subspaces or $k$-dimensional subspaces (called
\textit{blocks}), such that each $t$-subspace of $\ensuremath{\mathbb{F}}_{q}^{n}$
is contained in at most $\lambda$ codewords of $\mathcal{C}$.
\end{defn}

\section{Theorem}
\begin{thm}
If $n,k,t,$ and $\lambda$ are positive integers such that $1\leq t<k<n$
and $1\leq\lambda\leq\left[\begin{array}{c}
n-t\\
k-t
\end{array}\right]_{q}$, then

\[
\mathcal{A}_{q}\left(n,k,t;\lambda\right)\leq\left\lfloor \lambda\frac{\left[\begin{array}{c}
n\\
t
\end{array}\right]_{q}}{\left[\begin{array}{c}
k\\
t
\end{array}\right]_{q}}\right\rfloor 
\]
\end{thm}
%
\begin{thm}
If $n,k,t,$ and $\lambda$ are positive integers such that $1\leq t<k<n$
and $1\leq\lambda\leq\left[\begin{array}{c}
n-t\\
k-t
\end{array}\right]_{q}$, then

\[
\mathcal{A}_{q}\left(n,k,t;\lambda\right)\leq\left\lfloor \frac{q^{n}-1}{q^{k}-1}\mathcal{A}_{q}\left(n-1,k-1,t-1;\lambda\right)\right\rfloor 
\]
\end{thm}

\clearpage