%%%%%%%%%%%%%%%%%%%%%%%%%%%%%%%%%%%%%%%%%%%%%%%
\chapter{Preliminaries} \label{chap:preliminaries}
%%%%%%%%%%%%%%%%%%%%%%%%%%%%%%%%%%%%%%%%%%%%%%%

\section{Notation and Basic Terminology}

\paragraph{Vectors and Matrices}

Vectors $\boldsymbol{v}$ are denoted by underlined letters. Unless
stated otherwise, vectors are indexed starting from 1, i.e. $\boldsymbol{v}=\left[v_{1},\ldots,v_{n}\right]$.
Vectors are usually considered to be row vectors. Matrices $\boldsymbol{V}$
are shown in bold and capital letters. Elements of matrices or vectors
are surrounded by square brackets, and elements of tuples are surrounded
by round brackets. Curly brackets are used to cover elements of sets,
otherwise they are clearly stated for any other uses.

\paragraph{Vector space}

A vector space of dimension $n$ over a finite field with $q$ elements
is denoted by $\ensuremath{\mathbb{F}}_{q}^{n}$. 

\paragraph{Gaussian coefficient}

Gaussian coefficient (also known as $q$-binomial) counts the number
of subspaces of dimension $k$ in a vector space $\ensuremath{\mathbb{F}}_{q}^{n}$,

\[
\left[\begin{array}{c}
n\\
k
\end{array}\right]_{q}=\stackrel[i=0]{k-1}{\prod}\frac{q^{n}-q^{i}}{q^{k}-q^{i}}
\]


\paragraph{Multigraph}

A graph is permitted to have multiple edges. Edges that are incident
to same nodes can be in parallel. 

\paragraph{Directed Acyclic Graph}

A finite directed graph with no directed cycles, i.e. it consists
of a finite number nodes and edges, with each edge directed from a
vertex to another, such that there is no loop from any vertex $v$
with a sequence of directed edges back to the vertex again $v$.

\paragraph{Multicast}

Multicast communication supports the distribution of a data packet
to a group of users \cite{Zhang:2012}. It can be one-to-many or many-to-many
distribution \cite{Harte:2008}. In this study, we consider only one-to-many
multicast network.

\paragraph{Asymptotic Behavior}

For the combinatorial results, we study the asymtotic behaviour of
some formulas depending on the alphabet size $q$ and the vector length
$t$, by using the Bachmann-Landau notation, i.e. $\mathcal{O}\left(f\left(q,t\right)\right)$
for upper, $\Theta\left(f\left(q,t\right)\right)$ for tight, and
$\Omega\left(f\left(q,t\right)\right)$ for lower bounds, where $f$
is a function of the alphabet size and the vector length.

\section{Definition}
\begin{defn}[Rank-metric code]
 A linear $\left[m\times n,k,\delta\right]_{q}^{R}$ rank-metric
code $\mathcal{C}$ is a $k$-dimensional subspace of $\ensuremath{\mathbb{F}}_{q}^{m\times n}$
with mimum rank distance $\delta$.
\end{defn}

\paragraph{Maximum Rank Distance (MRD) code}

Let $rk\left[\boldsymbol{V}\right]$ be the rank of a matrix $\boldsymbol{V}\in\ensuremath{\mathbb{F}}_{q}^{m\times n}$.
The \textit{rank distance} between $\boldsymbol{U},\boldsymbol{V}\in\ensuremath{\mathbb{F}}_{q}^{m\times n}$
is defined by $d_{R}\left(\boldsymbol{U},\boldsymbol{V}\right)=rk\left[\boldsymbol{U}-\boldsymbol{V}\right]$
\cite{Delsarte:1978,Gabidulin:1985,Roth:1991}. The minimum rank distance
of a $\left[m\times n,k,\delta\right]_{q}^{R}$ rank-metric code $\mathcal{C}$
is defined by: $\delta=\underset{\boldsymbol{V}\in\mathcal{C},\boldsymbol{V}\neq\boldsymbol{0}}{min}\left\{ rk\left[\boldsymbol{V}\right]\right\} $.
Rank-metric codes that attained the Singleton-like upper bound $k\leq max\left\{ m,n\right\} \left(min\left\{ m,n\right\} -\delta+1\right)$
\cite{Delsarte:1978,Gabidulin:1985,Roth:1991} are called maximum
rank distance (MRD) codes and denoted by $\mathcal{MRD}\left[m\times n,\delta\right]_{q}$.
\begin{defn}[Grassmannian Code]
 A Grassmannian code is a set of all subspaces of dimension $k\leq n$
in $\ensuremath{\mathbb{F}}_{q}^{n}$, and is denoted by $\mathcal{G}_{q}\left(n,k\right)$.
Due to being the set of all subspaces that have the same dimension
$k$, it is also called a \textit{constant dimension code}. \cite{Zhang:2019}
\end{defn}
%
\begin{defn}[Projective Space]
 The \textit{projective space of order} $n$ is a set of all subspaces
of $\ensuremath{\mathbb{F}}_{q}^{n}$, and is denoted by $\mathcal{P}_{q}\left(n\right)$,
i.e. a union of all dimension $k=0,\ldots n$ subspaces in $\ensuremath{\mathbb{F}}_{q}^{n}$
or $\mathcal{P}_{q}\left(n\right)=\bigcup_{k=0}^{n}\mathcal{G}_{q}\left(n,k\right)$.
\cite{Wachter-Zeh:2018}
\end{defn}
%
\begin{defn}[Covering Grassmannian Code]
 An $\alpha-\left(n,k,\delta\right)_{q}^{c}$ covering Grassmannian
code (code in short) $\mathcal{C}$ is a subset of $\mathcal{G}_{q}\left(n,k\right)$
such that each subset of $\alpha$ codewords of $\mathcal{C}$ span
a subspace whose dimension is at least $\delta+k$ in $\ensuremath{\mathbb{F}}_{q}^{n}$.
\cite{Zhang:2019}
\end{defn}

\paragraph{The Cardinality of a Grassmannian Code}

The cardinality of $\mathcal{G}_{q}\left(n,k\right)$ is the Gaussian
coefficient (also known as $q$-binomial), which counts the number
of subspaces of dimension $k$ in a vector space $\ensuremath{\mathbb{F}}_{q}^{n}$,

\[
\left|\mathcal{G}_{q}\left(n,k\right)\right|=\left[\begin{array}{c}
n\\
k
\end{array}\right]_{q}=\stackrel[i=0]{k-1}{\prod}\frac{q^{n}-q^{i}}{q^{k}-q^{i}},
\]

where $q^{\left(n-k\right)k}\leq\left[\begin{array}{c}
n\\
k
\end{array}\right]_{q}\leq4q^{\left(n-k\right)k}$.

\begin{defn}[Multiple Grassmannian Code \cite{Etzion:2018}]
 A $\mathrm{t}-\left(n,k,\lambda\right)_{q}^{m}$ multiple Grassmannian
code, i.e. a subspace packing, is a set $\mathcal{S}$ of $k$-subspaces
or $k$-dimensional subspaces (called \textit{blocks}), such that
each $\mathrm{t}$-subspace of $\ensuremath{\mathbb{F}}_{q}^{n}$
is contained in at most $\lambda$ codewords of $\mathcal{C}$. 
\end{defn}

\paragraph*{Maximum Size of a Multiple Grassmannian Code}

$\mathcal{A}_{q}\left(n,k,\mathrm{t};\lambda\right)$ denotes the
maximum size of a $\mathrm{t}-\left(n,k,\lambda\right)_{q}^{m}$ code,
where there are no repeated codewords. \cite{Etzion:2018}

\clearpage